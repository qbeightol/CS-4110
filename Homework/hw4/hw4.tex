\documentclass[10pt, oneside]{article}
\usepackage{geometry}
\geometry{letterpaper}
\usepackage{amssymb, amsmath, enumerate, stmaryrd, bussproofs, graphicx}


\begin{document}
\noindent Ameya Acharya (apa52) \& Quinn Beightol (qeb2) \\*
\noindent CS 4110 \\*
\noindent Homework 4
\noindent Due Wednesday, October 1, 2014

\begin{enumerate}[1.]
% Exercise 1 %%%%%%%%%%%%%%%%%%%%%%%%%%%%%%%%%%%%%%%%%%%%%%%%%%%%%%%%%%%%%%%%%%%%%%%
  \item
  \begin{enumerate}[(a)] 
    \item 
    \begin{eqnarray*}
      \mathcal{C}\llbracket \texttt{if } b \texttt{ then } c \rrbracket & = & 
        \{ (\sigma, \sigma) \mid (\sigma, \texttt{false}) \in \mathcal{B}
                                                          \llbracket b \rrbracket \}
        \cup \{ (\sigma, \sigma') \mid (\sigma, \texttt{true}) \in \mathcal{B}
          \llbracket b \rrbracket \wedge (\sigma, \sigma') \in \mathcal{C}
                                                       \llbracket c \rrbracket \}\\
      \mathcal{C} \llbracket \texttt{do } c \texttt{ until } b  \rrbracket 
        & = & \text{fix}(G)\\
      \text{where } G(f) & = &
  				\{ (\sigma, \sigma') \mid (\sigma, \sigma') \in \mathcal{C} 
				                                              \llbracket c \rrbracket
		 \wedge (\sigma', \texttt{true}) \in \mathcal{B} \llbracket b \rrbracket \} \\
				& \cup & \{ (\sigma, \sigma'') \mid 
				            (\sigma, \sigma')  \in \mathcal{C} \llbracket c \rrbracket 
				     \wedge (\sigma', \texttt{false})\in \mathcal{B} \llbracket b \rrbracket
				     \wedge \exists \sigma''. (\sigma', \sigma'') \in f \}
    \end{eqnarray*}
    
    \item \text{ }
    \begin{prooftree}
      \AxiomC{$\vdash \{P \wedge b\} \text{ } c \text{ } \{Q\}$}
      \AxiomC{$\vdash \{P \wedge \neg b\} \Rightarrow Q$}
      \RightLabel{One-armed If}
      \BinaryInfC{$\vdash \{P\} \text{ if } b \text{ then } c \text{ }\{Q\}$}
    \end{prooftree}
    
    \begin{prooftree}
      \AxiomC{$\vdash \{P\} \text{ } c \text{ } \{Q\}$}
      \AxiomC{$\vdash \{Q \wedge \neg b\} \text{ } c \text{ } \{Q \}$}
      \RightLabel{Do}
      \BinaryInfC{$\vdash \{ P \} \texttt{ do } c \texttt{ until } b \{Q \wedge ^\neg b\}$}
    \end{prooftree}
  \end{enumerate}

% Exercise 2 %%%%%%%%%%%%%%%%%%%%%%%%%%%%%%%%%%%%%%%%%%%%%%%%%%%%%%%%%%%%%%%%%%%%%%
  \item
  \begin{enumerate} [(a)]
  \item 
  To show prove this equivalence, we will show that both expressions denote the same mathematical object. To make this easier to read, we will box the sections that we denote in later lines.\\
  
{\bf Denotation of}  (x := x + 21; x := x + 21):\\

      \begin{eqnarray*}  \mathcal{C} \llbracket x := x + 21; x := x + 21 \rrbracket & = & \{ (\sigma, \sigma') \mid \exists \sigma''.  ((\sigma, \sigma'') \in \boxed {\mathcal{C} \llbracket x := x + 21\rrbracket} \wedge (\sigma'', \sigma') \in   {\mathcal{C} \llbracket x := x + 21\rrbracket })\}   \\
    \mathcal{C} \llbracket x := x + 21\rrbracket & = & \{ (\sigma, \sigma[x \mapsto n]) \mid (\sigma, n) \in \boxed{\mathcal{A} \llbracket x + 21 \rrbracket} \} \\
     \mathcal{A} \llbracket x + 21\rrbracket & = & \{ (\sigma, n) \mid (\sigma, n_1) \in \boxed{\mathcal{A} \llbracket x \rrbracket} \wedge (\sigma, n_2) \in \boxed{\mathcal{A} \llbracket 21 \rrbracket} \wedge n = n_1 + n _2\} \\
       \mathcal{A} \llbracket x \rrbracket  & = & \{ (\sigma, \sigma(x)) \} \\
    \mathcal{A} \llbracket 21 \rrbracket  & = & \{ (\sigma, 21) \} \\
   \text{So:} \hspace{1mm} \mathcal{A} \llbracket x + 21 \rrbracket & = & \{ (\sigma, n) \mid (\sigma, n_1) \in (\sigma, \sigma(x)) \wedge (\sigma, n_2) \in (\sigma, 21) \wedge n = n_1 + n_2 \} \\
    & = & \{ (\sigma, n) \mid n = \sigma(x) + 21 \} \\
    & = & \{ (\sigma, \sigma(x) + 21) \} \\
 \text{And now:} \hspace{1mm} \mathcal{C} \llbracket x := x + 21\rrbracket & = & \{ (\sigma, \sigma[x \mapsto \sigma(x) + 21]) \} \\
  \end{eqnarray*}
  Now that we have found the denotation of the first ${\mathcal{C} \llbracket x := x + 21\rrbracket }$, we may use that in our   $\mathcal{C} \llbracket x := x + 21; x := x + 21 \rrbracket$:
  
    \begin{eqnarray*}   \mathcal{C} \llbracket x := x + 21; x := x + 21 \rrbracket & = & \{ (\sigma, \sigma') \mid \exists \sigma''.  ((\sigma, \sigma'') \in  \{ (\sigma, \sigma[x \mapsto \sigma(x) + 21]) \} \wedge (\sigma'', \sigma') \in \\ \mathcal{C} \llbracket x := x + 21\rrbracket )\}\\
      \mathcal{C} \llbracket x := x + 21; x := x + 21 \rrbracket & = & \{ (\sigma, \sigma') \mid \exists \sigma''.  \sigma'' = \sigma[x \mapsto \sigma(x) + 21] \} \wedge (\sigma'', \sigma') \in \boxed{\mathcal{C} \llbracket x := x + 21\rrbracket} )\}\\
       \mathcal{C} \llbracket x := x + 21\rrbracket & = & \{ (\sigma'', \sigma''[x \mapsto n]) \mid (\sigma'', n) \in \boxed{\mathcal{A} \llbracket x + 21 \rrbracket} \} \\
     \mathcal{A} \llbracket x + 21\rrbracket & = & \{ (\sigma'', n) \mid (\sigma'', n_1) \in \boxed{\mathcal{A} \llbracket x \rrbracket }\wedge (\sigma'', n_2) \in \boxed{\mathcal{A} \llbracket 21 \rrbracket }\wedge n = n_1 + n _2\} \\
       \mathcal{A} \llbracket x \rrbracket  & = & \{ (\sigma'', \sigma''(x)) \} \\
         \mathcal{A} \llbracket x \rrbracket  & = & \{ (\sigma'', \sigma(x) + 21) \} \\
    \mathcal{A} \llbracket 21 \rrbracket  & = & \{ (\sigma'', 21) \} \\
     \text{So:} \hspace{1mm}  \mathcal{A} \llbracket x + 21 \rrbracket & = & \{ (\sigma'', n) \mid (\sigma'', n_1) \in (\sigma'', \sigma''(x)) \wedge (\sigma'', n_2) \in (\sigma'', 21) \wedge n = n_1 + n_2 \} \\
    & = & \{ (\sigma'', n) \mid n = \sigma(x) + \sigma(x) + 21 \} \\
    & = & \{ (\sigma'', \sigma(x) + 42) \} \\
       \text{Now, we know:} \hspace{1mm}  \mathcal{C} \llbracket x := x + 21\rrbracket & = & \{ (\sigma'', \sigma''[x \mapsto n]) \mid (\sigma'', n) \in \{ (\sigma'', \sigma(x) + 42) \} \} \\
     \text{So, we conclude:}\\ \mathcal{C} \llbracket x := x + 21; x := x + 21 \rrbracket & = & \{(\sigma, \sigma[x \mapsto \sigma(x) + 42 ]) \}
    \end{eqnarray*}

{\bf Denotation of}  (x := x + 42):
      \begin{eqnarray*}  \mathcal{C} \llbracket x := x + 42 \rrbracket & = & \{ (\sigma, \sigma[x \mapsto n]) \mid (\sigma, n) \in \boxed{\mathcal{A} \llbracket x + 42 \rrbracket \} }\\
    \mathcal{A} \llbracket x + 42 \rrbracket & = & \{ (\sigma, n) \mid (\sigma, n_1) \in \boxed{\mathcal{A} \llbracket x \rrbracket} \wedge (\sigma, n_2) \in \boxed{\mathcal{A} \llbracket 42 \rrbracket }\wedge n = n_1 + n _2\} \\
    \mathcal{A} \llbracket x \rrbracket  & = & \{ (\sigma, \sigma(x)) \} \\
    \mathcal{A} \llbracket 42 \rrbracket  & = & \{ (\sigma, 42) \} \\
       \text{So:} \hspace{1mm} \mathcal{A} \llbracket x + 42 \rrbracket & = & \{ (\sigma, n) \mid (\sigma, n_1) \in (\sigma, \sigma(x)) \wedge (\sigma, n_2) \in (\sigma, 42) \wedge n = n_1 + n_2 \} \\
    & = & \{ (\sigma, n) \mid n = \sigma(x) + 42 \} \\
    & = & \{ (\sigma, \sigma(x) + 42) \} \\
     \mathcal{C} \llbracket x := x + 42 \rrbracket & = & \{ (\sigma, \sigma[x \mapsto n]) \mid n = \sigma(x) + 42 \} \\
     & = &  \{ (\sigma, \sigma[x \mapsto \sigma(x) + 42 ])\} 
   \end{eqnarray*}
   
   The two are the same.
   \pagebreak
   
% Exercise 2 b %%%%%%%%%%%%%%%%%%%%%%%%%%%%%%%%%%%%%%%%%%%%%%%%%%%%%%%%%%%%%%%%%%%%
    \item
      To conserve space, we'll refer to 
      $(x \texttt{:= }1\texttt{; do } x \texttt{:= } x + 1 \texttt{ until } x < 0)$ 
      as $c_a$, and we'll refer to $\texttt{while true do } c$ as $c_b$. 
      
      \begin{eqnarray*}
        \mathcal{C} \llbracket c_a \rrbracket & = & 
          \{(\sigma, \sigma'') \mid \exists \sigma'. 
            (\sigma, \sigma') \in \mathcal{C}\llbracket x\texttt{:= } 1\rrbracket \\
     & & \wedge (\sigma', \sigma'') \in \mathcal{C} \llbracket \texttt{do } x 
                               \texttt{:= } x + 1 \texttt{ until } x < 0 \rrbracket
                                   \} \\
        \mathcal{C} \llbracket x \texttt{:= } x+1 \rrbracket & = & \{(\sigma, n) \mid (\sigma, \sigma[x \mapsto n]) \in \mathcal{A} \llbracket x \rrbracket  \} \\
        \mathcal{A} \llbracket 1 \rrbracket & = & \{(\sigma, 1) \} \\
        \mathcal{C} \llbracket x \texttt{:= } x+1 \rrbracket & = & \{ (\sigma, \sigma[x \mapsto 1]) \} \\
        & & \\
        \mathcal{C} \llbracket \texttt{do } x \texttt{:= } x + 1 \texttt{ until } x < 0 \rrbracket & = & \text{fix}(G') \\
        \text{where } G'(f) & = & \{ (\sigma, \sigma') \mid (\sigma, \sigma') \in \mathcal{C} \llbracket x \texttt{:= } x + 1 \rrbracket \wedge (\sigma', \texttt{true}) \in \mathcal{B} \llbracket x < 0 \rrbracket \} \\
				& \cup & \{ (\sigma, \sigma'') \mid (\sigma, \sigma')  \in \mathcal{C} \llbracket x \texttt{:= } x + 1 \rrbracket \wedge (\sigma', \texttt{false})\in \mathcal{B} \llbracket x < 0 \rrbracket \\
				& & \wedge \exists \sigma''. (\sigma', \sigma'') \in f \}
      \end{eqnarray*}
      
      Going forward, we'll argue that the second set in the definition of $G'$ 
      (more precisely, the set defined after the first union operator) never adds 
      anything to $f_n$--i.e. we'll prove $\forall n \in \mathbb{N}. 
      P(n) = \{ (\sigma, \sigma'') \mid ... \wedge \exists \sigma''. (\sigma', 
      \sigma'') \in f_n \} = \emptyset$ using induction: \\
      
      \underline{P(0)}: In this case $f_n = \emptyset$ so its impossible to have
      a pair $(\sigma', \sigma'')$ satisfying $(\sigma', \sigma'')$, so the the 
      second set is the empty set. \\
      
      \underline{Inductive Step (i.e. $P(n - 1) \rightarrow P(n)$)}: Using the 
      inductive hypothesis, we conclude that 
      
      \begin{eqnarray*}
        f_n = & = & \{ (\sigma, \sigma') \mid (\sigma, \sigma') \in \mathcal{C} \llbracket x \texttt{:= } x + 1 \rrbracket \wedge (\sigma', \texttt{true}) \in \mathcal{B} \llbracket x < 0 \rrbracket \}  \\
        & \cup & \{ (\sigma, \sigma'') \mid (\sigma, \sigma')  \in \mathcal{C} \llbracket x \texttt{:= } x + 1 \rrbracket \wedge (\sigma', \texttt{false})\in \mathcal{B} \llbracket x < 0 \rrbracket \\
				& & \wedge \exists \sigma''. (\sigma', \sigma'') \in f_{n-1} \} \\
				& = & \{ (\sigma, \sigma') \mid (\sigma, \sigma') \in \mathcal{C} \llbracket x \texttt{:= } x + 1 \rrbracket \wedge (\sigma', \texttt{true}) \in \mathcal{B} \llbracket x < 0 \rrbracket \}  \\
        & \cup & \{ (\sigma, \sigma'') \mid (\sigma, \sigma')  \in \mathcal{C} \llbracket x \texttt{:= } x + 1 \rrbracket \wedge (\sigma', \texttt{false})\in \mathcal{B} \llbracket x < 0 \rrbracket \\
				& & \wedge \exists \sigma''. (\sigma', \sigma'') \in \emptyset \} \\
			\end{eqnarray*}
			\begin{eqnarray*}
				f_n & = & \{ (\sigma, \sigma') \mid (\sigma, \sigma') \in \mathcal{C} \llbracket x \texttt{:= } x + 1 \rrbracket \wedge (\sigma', \texttt{true}) \in \mathcal{B} \llbracket x < 0 \rrbracket \}  \\
			\end{eqnarray*}
			
			\pagebreak
			And now we can re-write the second set in $G'(f)$:
			
			\begin{eqnarray*}
			  \{ (\sigma, \sigma'') & \mid & (\sigma, \sigma')  \in \mathcal{C} \llbracket x \texttt{:= } x + 1 \rrbracket \\ 
			  & \wedge & (\sigma', \texttt{false})\in \mathcal{B} \llbracket x < 0 \rrbracket \\
				& \wedge & \exists \sigma''. (\sigma', \sigma'') \in f_n \} \\
				& & \\
				= \{ (\sigma, \sigma'') & \mid & (\sigma, \sigma')  \in \mathcal{C} \llbracket x \texttt{:= } x + 1 \rrbracket \\ 
			  & \wedge & (\sigma', \texttt{false})\in \mathcal{B} \llbracket x < 0 \rrbracket \\
				& \wedge & \exists \sigma''. ((\sigma', \sigma'') \in \mathcal{C} \llbracket x \texttt{:= } x + 1 \rrbracket \\
				& \wedge & (\sigma'', \texttt{true}) \in \mathcal{B} \llbracket x < 0 \rrbracket) \}
			\end{eqnarray*}
			
			Notice that the last three requirements on the stores are contradictory. 
			Consider some $\sigma'$ such that $\mathcal{B} \llbracket x < 0 \rrbracket 
			\sigma' = \texttt{false}$.  Looking at at the definition of 
			$\mathcal{B} \llbracket x < 0 \rrbracket$ we can infer a few facts about 
			$\sigma'$:
			
			\begin{eqnarray*}
			  \mathcal{B} \llbracket x < 0 \rrbracket & = & ... \cup \{ (\sigma', \texttt{false}) \mid (\sigma', n_1) \in \mathcal{A} \llbracket x \rrbracket \wedge (\sigma', n_2) \in \mathcal{A} \llbracket 0 \rrbracket \wedge n_1 \geq n_2 \} \\
			  \mathcal{A} \llbracket x \rrbracket & = & \{(\sigma, \sigma(x))\} \\
			  \mathcal{A} \llbracket 0 \rrbracket & = & \{(\sigma, 0)\} \\
			\end{eqnarray*}
			
			Since $\mathcal{B} \llbracket x < 0 \rrbracket \sigma' = \texttt{true}$, we
			know that $n_1 = \sigma'(x)$, $n_2 = 0$, and $\sigma'(x) \geq 0$. Now let's 
			take a look at what happens when execute $x \texttt{:= } x + 1$ on the 
			store and then evaluate $x < 0$:
			
			\begin{eqnarray*}
			  \mathcal{C} \llbracket x \texttt{:= } x + 1 \rrbracket & = & \{ (\sigma, \sigma[x \mapsto n]) \mid (\sigma, n) \in \mathcal{A} \llbracket x + 1 \rrbracket \} \\
			  \mathcal{A} \llbracket x + 1 \rrbracket & = & \{(\sigma, n') \mid (\sigma, n_1') \in \mathcal{A} \llbracket x \rrbracket \wedge (\sigma, n_2') \in \mathcal{A} \llbracket x \rrbracket \wedge n' = n_1' + n_2' \}
			\end{eqnarray*}
			
			\begin{eqnarray*}
			  \mathcal{A} \llbracket x \rrbracket & = & \{(\sigma, \sigma(x))\} \\
			  \mathcal{A} \llbracket 1 \rrbracket & = & \{(\sigma, 1)\} \\
			  \mathcal{A} \llbracket x + 1 \rrbracket & = & \{(\sigma, \sigma(x) + 1)\} \\
			  \mathcal{C} \llbracket x \texttt{:= } x + 1 \rrbracket & = & \{(\sigma, \sigma[x \mapsto \sigma(x) + 1])\} \\
			  \mathcal{B} \llbracket x < 0 \rrbracket & = & \{(\sigma, \texttt{true}) \mid (\sigma, n_1) \in \mathcal{B} \llbracket x \rrbracket \wedge (\sigma, n_2) \in \mathcal{B} \llbracket 0 \rrbracket \wedge n_1 < n_2\} \cup \\
			  & & \{(\sigma, \texttt{false}) \mid (\sigma,n_1) \in \mathcal{B} \llbracket x \rrbracket \wedge (\sigma, n_2) \in \mathcal{B} \llbracket 0 \rrbracket \wedge n_1 \geq n_2 \}
			\end{eqnarray*}
			
			which can be more compactly written:
			
			$$\mathcal{B} \llbracket x < 0 \rrbracket = \left \{ 
			  \begin{array}{lr} 
			    \texttt{true}  & \text{iff } \sigma'(x) < 0 \\
			    \texttt{false} & \text{otherwise}
			  \end{array} \right.$$
			    
		  Now the source of the contradiction should become clear. Let's evaluate
		  our command on $\sigma'$ and then evaluate our boolean expression:
			\begin{eqnarray*}
			  \sigma'' = \mathcal{C} \llbracket x \texttt{:= } x + 1 \rrbracket \sigma' & = & \sigma'[x \mapsto \sigma'(x) + 1] \\
			  \mathcal{B} \llbracket x < 0 \rrbracket \sigma'' = \texttt{false}
			\end{eqnarray*}
			
			The boolean expression evaluates to false because $\sigma'(x) > 0$ and so 
			$\sigma''(x) = \sigma'(x) + 1 > 0$. This, in turn, demonstrates that the 
			three conditionals in the definition of the second set are contradictory. 
			That set will be the empty set for all $n$. \checkmark
			
			
			Now we can calculate the fixed point of $G'$:
			
			\begin{eqnarray*}
			  \text{fix}(G') & = & \bigcup_{i \geq 0} (G')^{i}(\emptyset) \\
			  & = & \emptyset \cup \{ (\sigma, \sigma') \mid (\sigma, \sigma') \in \mathcal{C} \llbracket x \texttt{:= } x + 1 \rrbracket \wedge (\sigma', \texttt{true}) \in \mathcal{B} \llbracket x < 0 \rrbracket \} \cup \emptyset \cup\\
			  & & \{ (\sigma, \sigma') \mid (\sigma, \sigma') \in \mathcal{C} \llbracket x \texttt{:= } x + 1 \rrbracket \wedge (\sigma', \texttt{true}) \in \mathcal{B} \llbracket x < 0 \rrbracket \} \cup \emptyset \cup ... \\
			  & = & \{ (\sigma, \sigma') \mid (\sigma, \sigma') \in \mathcal{C} \llbracket x \texttt{:= } x + 1 \rrbracket \wedge (\sigma', \texttt{true}) \in \mathcal{B} \llbracket x < 0 \rrbracket \} \\
			\end{eqnarray*}
			
			If we evaluate $\mathcal{C} \llbracket x \texttt{:= 1} \rrbracket$ on a store
			sigma, we'll get a new store $\sigma' = \sigma[x\mapsto 1]$, and then if we 
			attempt to apply  $\mathcal{C} \llbracket \texttt{do } x \texttt{:= } x + 1 
			\texttt{ until } x < 0 \rrbracket  =  \text{fix}(G')$ to that store, well, 
			we won't get a value--the relationship is undefined since $\sigma'(x) > 
			0$, which implies that $\sigma''(x) = \sigma'(x) + 1 > 0$ and therefore $
			\mathcal{B}\llbracket x < 0 \rrbracket \sigma'' = \texttt{false}$. In other 
			words, $c_a$ doesn't terminate. Let's look at $c_b$ and prove that it doesn't 
			terminate as well (which in turn demonstrates that the commands are 
			equivalent:
			
      \begin{eqnarray*}
        \mathcal{C} \llbracket c_b \rrbracket & = & \text{fix}(F) \\
        F(f) & = & \{ (\sigma, \sigma) \mid (\sigma, \texttt{false}) \in 
                                           \mathcal{B} \llbracket b \rrbracket \}\\
            & \cup & \{(\sigma, \sigma'') 
               \mid (\sigma, \texttt{true}) \in \mathcal{B} \llbracket b\rrbracket
             \wedge (\sigma, \sigma') \in \mathcal{C} \llbracket c \rrbracket 
             \wedge (\sigma', \sigma'') \in f \} \\
      \end{eqnarray*}
      Substituting \texttt{true} in for $b$ give the following equation (which is 
      only true for commands with the same form as $c_b$):
      \begin{eqnarray*}
        F'(f) & = & \{ (\sigma, \sigma) \mid (\sigma, \texttt{false}) \in 
                               \mathcal{B} \llbracket \texttt{true} \rrbracket \}\\
            & \cup & \{(\sigma, \sigma'') 
    \mid (\sigma, \texttt{true}) \in \mathcal{B} \llbracket \texttt{true}\rrbracket
             \wedge (\sigma, \sigma') \in \mathcal{C} \llbracket c \rrbracket 
             \wedge (\sigma', \sigma'') \in f \} \\
      \end{eqnarray*} 
      Since $\mathcal{B} \llbracket \texttt{true} \rrbracket = \{(\sigma, 
      \texttt{true})\}$, the pair $(\sigma, \texttt{false})$ will never be a subset 
      of $\mathcal{B} \llbracket \texttt{true} \rrbracket$ and the set preceding 
      the union operator will be the empty set. Likewise, $(\sigma, \texttt{true})$ 
      is always a subset of $\{(\sigma, \texttt{true})\}$, so we can refine our 
      definition of $F'$ as follows:
            \begin{eqnarray*}
        F'(f) & = &  \{(\sigma, \sigma'') 
    					    \mid (\sigma, \sigma') \in \mathcal{C} \llbracket c \rrbracket 
                \wedge (\sigma', \sigma'') \in f \} \\
      \end{eqnarray*} 
      Now we'll show that $f_n$ is undefined on all stores for all $n \in
      \mathbb{N}$ using induction (and the higher order function $F'$):
      \begin{eqnarray*}
        P(n)  & \triangleq & f_n = \emptyset \\
        P(0): & & f_0 = F^0(\emptyset) = \emptyset \text{ } \checkmark \\
        P(n+1): & & f_{n+1} = F(f_n) \\
                & = & \{(\sigma, \sigma'') 
    					    \mid (\sigma, \sigma') \in \mathcal{C} \llbracket c \rrbracket 
                \wedge (\sigma', \sigma'') \in f_n \}
      \end{eqnarray*}
      Using our inductive hypothesis, we can conclude that $f_n = \emptyset$ which
      in turn means that the second rule in the set comprehension (e.g. $(\sigma',
      \sigma'') \in f_n$) is unsatisfiable. So the right hand side of the equation
      is just the empty set, as desired:
      
      $$f_{n+1} = \emptyset \text{ } \checkmark$$ 
      
      Therefore, $P(n)$ hold for all $n \in \mathbb{N}$ and the fixed point of $F'$ 
      is:
      
      $$ \text{fix}(F') = \bigcup_{i \geq 0} F'^{i}(\emptyset) = \emptyset \cup 
      \emptyset \cup ... \cup \emptyset = \emptyset$$

      In short, $c_b$ is guaranteed not to terminate, just like $c_a$.
      
    \item {\bf Denotation of}  (x := x):
       \begin{eqnarray*}  \mathcal{C} \llbracket x := x \rrbracket & = & \{ (\sigma, \sigma[x \mapsto n]) \mid (\sigma, n) \in \mathcal{A} \llbracket x \rrbracket \} \\
    \mathcal{A} \llbracket x \rrbracket & = & \{ (\sigma, \sigma)\} \\
     \mathcal{C} \llbracket x := x \rrbracket & = & \{ (\sigma, \sigma[x \mapsto \sigma(x)]) \}\\
     & = & \{(\sigma, \sigma)\}\\
   \end{eqnarray*}
   
   
   {\bf Denotation of}   ($\texttt{if} (x = x+1) \hspace{1mm} \texttt{then} \hspace{1mm} x := 0$):
       \begin{eqnarray*}  \mathcal{C} \llbracket \texttt{if} (x = x+1) \hspace{1mm} \texttt{then}\hspace{1mm} x := 0 \rrbracket & = & \{ (\sigma, \sigma) \mid (\sigma, \texttt{false}) \in \mathcal{B} \llbracket x = x + 1 \rrbracket \} \cup \{ (\sigma, \sigma') \mid (\sigma, \texttt{true}) \in \mathcal{B}\\  \llbracket x = x + 1 \rrbracket \wedge (\sigma, \sigma') \in \mathcal{C}\llbracket x := 0 \rrbracket \} \\
    \mathcal{A} \llbracket x \rrbracket & = & \{ (\sigma, \sigma)\} \\
     \mathcal{C} \llbracket x := x \rrbracket & = & \{ (\sigma, \sigma[x \mapsto \sigma(x)]) \}\\
     & = & \{(\sigma, \sigma)\}
   \end{eqnarray*}
   
   The two are the same.

  \end{enumerate}

% Exercise 3 %%%%%%%%%%%%%%%%%%%%%%%%%%%%%%%%%%%%%%%%%%%%%%%%%%%%%%%%%%%%%%%%%%%%%%%
  \item Our loop invariant is: $r = n - qm \wedge q \geq 0.$

% Exercise 4 %%%%%%%%%%%%%%%%%%%%%%%%%%%%%%%%%%%%%%%%%%%%%%%%%%%%%%%%%%%%%%%%%%%%%%
\pagebreak

    \item     \hspace{4mm}
    
    \includegraphics[scale = 0.72]{/Users/ameyaacharya/Desktop/4a.png}
    \item


% Exercise 5 %%%%%%%%%%%%%%%%%%%%%%%%%%%%%%%%%%%%%%%%%%%%%%%%%%%%%%%%%%%%%%%%%%%%%%%
  \item
  \begin{enumerate} [(a)]
    \item
      Yes, consider the following derivation tree (as a side note, we'll refer to 
  		$(b \Rightarrow P) \wedge (\neg b \Rightarrow Q)$ as A:
      
      \begin{prooftree}
        \AxiomC{$\vDash \text{A}  \Rightarrow \text{A}$}
        
        \AxiomC{$\vDash (b \Rightarrow P) \wedge (\neg b \Rightarrow Q) \wedge b \Rightarrow P$}
        \AxiomC{$\vdash \{\text{P}\} \text{ } c \text{ } \{\text{A}\}$}
        \AxiomC{$\vDash A \Rightarrow A$}
        \LeftLabel{C}
        \TrinaryInfC{$\vdash \{(b \Rightarrow P) \wedge (\neg b \Rightarrow Q) \wedge b \} \text{ } c \text{ } \{A\}$}
        \LeftLabel{W}
        \UnaryInfC{$\vdash \{ \text{A} \} \texttt{ while b do c } \{ \text{A} \wedge \neg b \}$}
        
        \AxiomC{A $\wedge \neg b$ $\Rightarrow$ Q}
        \LeftLabel{C}
        \TrinaryInfC{$\vdash \{\text{A}\} \texttt{ while b do c } \{\text{Q}\}$}
      \end{prooftree}
      
      *$"C"$ is an abbreviation for "Consequence", and $"W"$ for "While" \\*
      
      Note that we've made a few assumptions here, namely that we have some 
      derivation for $\{\text{P}\} \text{ } c \text{ } \{\text{A}\}$ as well as 
      A $\Rightarrow$ A, A $\Rightarrow$ Q, and $(b \Rightarrow P) \wedge (\neg b 
      \Rightarrow Q) \wedge b \Rightarrow P$. The last three assumptions are 
      tautologies, however--anything implies itself, and we can combine $(b 
      \Rightarrow P) \wedge (\neg b \Rightarrow Q)$ with information about the
      truth value of $b$ to conclude either $P$ (if $b$ is true) or $Q$ (if $b$ 
      is false). Consequently, the entire tree is a proof that 
      
      \begin{prooftree}
        \AxiomC{$\vdash \{P\} \text{ } c \text{ } \{(b \Rightarrow P) \wedge (\neg b \Rightarrow Q) \}$}
        \UnaryInfC{$\vdash \{(b \Rightarrow P) \wedge (\neg b \Rightarrow Q)\} \texttt{ while b do c } \{Q\}$}
      \end{prooftree}
      
      which is the definition of our alternate \texttt{while} rule.
      
      
    \item 
      No, this rule reduces the completeness of Hoare logic. Consider the following
      code:
      
      \texttt{sum:= 0} \\
      \texttt{i:= 0} \\
      \texttt{while i < n do\{} \\
      \indent \texttt{ } \texttt{ i:= i + 1} \\ 
      \indent \texttt{ } \texttt{ sum:= sum + i} \\
      \texttt{\}} \\
      
      It computes $\Sigma_{j=1}^{n}j$, and it would be nice to be able to prove 
      that property using Hoare Logic. Put simply, you can't prove that property 
      using the new rule (primarily because you can only show that the sum is $
      \Sigma_{j=1}^{i}j$ and that $i <= n+1$, but not that $i <=n$), but we'll 
      delve into a slightly longer analysis to show where the bug occurs. Here's 
      part of the proof we'd write using the original inference rules (along with 
      a few labels for lengthier subsections of the tree: \\*
      
      
      
      \underline{LT:} 
      \begin{prooftree}
        \AxiomC{}
        \LeftLabel{Assign}
        \UnaryInfC{$\vdash \{\texttt{sum} = \Sigma_{j=1}^{i}j \wedge i \leq n ^ i < n \} \texttt{i:= i + 1} \{\texttt{sum} = \Sigma_{j=1}^{i-1}j \wedge i - 1 \leq n \wedge i - 1 < n \}$} 
      \end{prooftree} 
      
      \underline{RT:} 
      \begin{prooftree}
      	\AxiomC{}
	    	\LeftLabel{Assign}
				\UnaryInfC{$\vdash \{\texttt{sum} = \Sigma_{j=1}^{i-1}j \wedge i - 1 \leq n \wedge i - 1 < n \} \texttt{sum:= sum + i} \{\texttt{sum} - i = \Sigma_{j=1}^{i-1} \wedge i - 1 \leq n \wedge i-1 < n\}$}
			\end{prooftree}
			
			\underline{A:} $\{\texttt{sum} = \Sigma_{j=1}^{i}j \wedge i \leq n \wedge i < n \}$
      
      \underline{B':} $\{\texttt{sum} - i = \Sigma_{j=1}^{i-1} \wedge i - 1 \leq n \wedge i-1 < n\}$
      
      \underline{B:} $\{ \texttt{sum} = \Sigma_{j=1}^{i}j \wedge i \leq n \}$
      
      Note that you can use arithmetic properties to conclude that B' $\Rightarrow$ 
      B
      
      \begin{prooftree}
        \AxiomC{}
        \LeftLabel{anything implies itself}
        \UnaryInfC{A $\Rightarrow$ A}
        \AxiomC{LT}
        \AxiomC{RT}
        \LeftLabel{Seq}
        \BinaryInfC{$ \{\text{A}\} \texttt{ i:=i+1; sum:= sum + i } \{ \text{B'} \}$}
        
        \AxiomC{}
        \LeftLabel{arithmetic}
        \UnaryInfC{B' $\Rightarrow$ B}
        \LeftLabel{Consequence}
        \TrinaryInfC{$\{ \text{A} \} \texttt{ i:=i+1; sum:= sum + i } \{ \text{B}\}$} 
        \LeftLabel{While}
        \UnaryInfC{$\{ \texttt{sum} - \Sigma_{j=1}^{i}j \wedge i \leq n \} \texttt{ while i < n do (i:= i + 1; sum:= sum + i)} \{ \text{B} \wedge i \geq n\} $}
      \end{prooftree}
      
      Notice how important it is for $j<n$ to be on the left-side of 
      \texttt{ i = i+1; sum := sum + i} above the while inference. Using that 
      information we can conclude that $\texttt{sum} = \Sigma_{j=1}^{i}j \wedge i 
      \leq n$ after executing the body of the loop (otherwise it might be possible 
      for $i = n +1$, which would only show that the final value of \texttt{sum} to 
      be either $\Sigma_{j=0}^{i}j \wedge (i = n \vee i = n + 1)$).  We need to 
      have $\{ P \wedge b \}$ at the top-left of our
      while rule for the language to be as complete as possible.
      
      
  \end{enumerate}

\end{enumerate}
\end{document}