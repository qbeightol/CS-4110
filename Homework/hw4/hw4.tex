\documentclass[10pt, oneside]{article}
\usepackage{geometry}
\geometry{letterpaper}
\usepackage{amssymb, amsmath, enumerate, stmaryrd}


\begin{document}
\noindent Ameya Acharya (apa52) \& Quinn Beightol (qeb2) \\*
\noindent CS 4110 \\*
\noindent hw4

\begin{enumerate}[1.]
% Exercise 1 %%%%%%%%%%%%%%%%%%%%%%%%%%%%%%%%%%%%%%%%%%%%%%%%%%%%%%%%%%%%%%%%%%%%%%%
  \item
  \begin{enumerate}[(a)] 
    \item 
    \begin{eqnarray*}
      \mathcal{C}\llbracket \texttt{if } b \texttt{ then } c \rrbracket & = & 
        \{ (\sigma, \sigma) \mid (\sigma, \texttt{false}) \in \mathcal{B}
                                                          \llbracket b \rrbracket \}
        \cup \{ (\sigma, \sigma) \mid (\sigma, \texttt{false}) \in \mathcal{B}
          \llbracket b \rrbracket \wedge (\sigma, \sigma') \in \mathcal{C}
                                                        \llbracket c \rrbracket \}\\
      \mathcal{C} \llbracket \texttt{do } c \texttt{ until } b  \rrbracket 
        & = & \text{fix}(G)\\
      \text{where } G(f) & = &
  				\{ (\sigma, \sigma') \mid (\sigma, \sigma') \in \mathcal{C} 
				                                              \llbracket c \rrbracket
		 \wedge (\sigma', \texttt{true}) \in \mathcal{B} \llbracket b \rrbracket \} \\
				& \cup & \{ (\sigma, \sigma'') \mid 
				            (\sigma, \sigma')  \in \mathcal{C} \llbracket c \rrbracket 
				     \wedge (\sigma', \texttt{false})\in \mathcal{B} \llbracket b \rrbracket
				     \wedge \exists \sigma''. (\sigma', \sigma'') \in f \}
    \end{eqnarray*}
    \item 
  \end{enumerate}

% Exercise 2 %%%%%%%%%%%%%%%%%%%%%%%%%%%%%%%%%%%%%%%%%%%%%%%%%%%%%%%%%%%%%%%%%%%%%%%
  \item
  \begin{enumerate} [(a)]

    \item  \begin{eqnarray*}  \mathcal{C} \llbracket x := x + 42 \rrbracket & = & \{ (\sigma, \sigma[x \mapsto n]) \mid (\sigma, n) \in \mathcal{A} \llbracket x + 42 \rrbracket \} \\
    \mathcal{A} \llbracket x + 42 \rrbracket & = & \{ (\sigma, n) \mid (\sigma, n_1) \in \mathcal{A} \llbracket x \rrbracket \wedge (\sigma, n_2) \in \mathcal{A} \llbracket 42 \rrbracket \wedge n = n_1 + n _2\} \\
    \mathcal{A} \llbracket x \rrbracket  & = & \{ (\sigma, \sigma(x)) \} \\
    \mathcal{A} \llbracket 42 \rrbracket  & = & \{ (\sigma, 42) \} \\
    \mathcal{A} \llbracket x + 42 \rrbracket & = & \{ (\sigma, n) \mid (\sigma, n_1) \in (\sigma, \sigma(x)) \wedge (\sigma, n_2) \in (\sigma, 42) \wedge n = n_1 + n_2 \} \\
    & = & \{ (\sigma, n) \mid n = \sigma(x) + 42 \} \\
    & = & \{ (\sigma, \sigma(x) + 42) \} \\
    & = & \{ (\sigma, \sigma(x) + 42 \}\\
     \mathcal{C} \llbracket x := x + 42 \rrbracket & = & \{ (\sigma, \sigma[x \mapsto n]) \mid n = \sigma(x) + 42 \} \\
     & = &  \{ (\sigma, \sigma[x \mapsto \sigma(x) + 42 ]\} \\
   \end{eqnarray*}
    \item
    \item
  \end{enumerate}

% Exercise 3 %%%%%%%%%%%%%%%%%%%%%%%%%%%%%%%%%%%%%%%%%%%%%%%%%%%%%%%%%%%%%%%%%%%%%%%
  \item

% Exercise 4 %%%%%%%%%%%%%%%%%%%%%%%%%%%%%%%%%%%%%%%%%%%%%%%%%%%%%%%%%%%%%%%%%%%%%%%
  \item
  \begin{enumerate} [(a)]
    \item
    \item
  \end{enumerate}

% Exercise 5 %%%%%%%%%%%%%%%%%%%%%%%%%%%%%%%%%%%%%%%%%%%%%%%%%%%%%%%%%%%%%%%%%%%%%%%
  \item
  \begin{enumerate} [(a)]
    \item
    \item
  \end{enumerate}

\end{enumerate}
\end{document}