\documentclass[10pt, oneside]{article}
\usepackage{geometry}
\geometry{letterpaper}
\usepackage{amssymb, amsmath, enumerate, stmaryrd}


\begin{document}
\noindent Ameya Acharya (apa52) \& Quinn Beightol (qeb2) \\*
\noindent CS 4110 \\*
\noindent hw4

\begin{enumerate}[1.]
% Exercise 1 %%%%%%%%%%%%%%%%%%%%%%%%%%%%%%%%%%%%%%%%%%%%%%%%%%%%%%%%%%%%%%%%%%%%%%%
  \item
  \begin{enumerate}[(a)] 
    \item 
    \begin{eqnarray*}
      \mathcal{C}\llbracket \texttt{if } b \texttt{ then } c \rrbracket & = & 
        \{ (\sigma, \sigma) \mid (\sigma, \texttt{false}) \in \mathcal{B}
                                                          \llbracket b \rrbracket \}
        \cup \{ (\sigma, \sigma') \mid (\sigma, \texttt{true}) \in \mathcal{B}
          \llbracket b \rrbracket \wedge (\sigma, \sigma') \in \mathcal{C}
                                                       \llbracket c \rrbracket \}\\
      \mathcal{C} \llbracket \texttt{do } c \texttt{ until } b  \rrbracket 
        & = & \text{fix}(G)\\
      \text{where } G(f) & = &
  				\{ (\sigma, \sigma') \mid (\sigma, \sigma') \in \mathcal{C} 
				                                              \llbracket c \rrbracket
		 \wedge (\sigma', \texttt{true}) \in \mathcal{B} \llbracket b \rrbracket \} \\
				& \cup & \{ (\sigma, \sigma'') \mid 
				            (\sigma, \sigma')  \in \mathcal{C} \llbracket c \rrbracket 
				     \wedge (\sigma', \texttt{false})\in \mathcal{B} \llbracket b \rrbracket
				     \wedge \exists \sigma''. (\sigma', \sigma'') \in f \}
    \end{eqnarray*}
    \item 
  \end{enumerate}

% Exercise 2 %%%%%%%%%%%%%%%%%%%%%%%%%%%%%%%%%%%%%%%%%%%%%%%%%%%%%%%%%%%%%%%%%%%%%%%
  \item
  \begin{enumerate} [(a)]
  \item 
{\bf Denotation of}  (x := x + 21; x := x + 21):
      \begin{eqnarray*}  \mathcal{C} \llbracket x := x + 21; x := x + 21 \rrbracket & = & \{ (\sigma, \sigma') \mid \exists \sigma''.  ((\sigma, \sigma'') \in \mathcal{C} \llbracket x := x + 21\rrbracket \wedge (\sigma'', \sigma') \in \mathcal{C} \llbracket x := x + 21\rrbracket )\}\\
    \mathcal{C} \llbracket x := x + 21\rrbracket & = & \{ (\sigma, \sigma[x \mapsto n]) \mid (\sigma, n) \in \mathcal{A} \llbracket x + 21 \rrbracket \} \\
     \mathcal{A} \llbracket x + 21\rrbracket & = & \{ (\sigma, n) \mid (\sigma, n_1) \in \mathcal{A} \llbracket x \rrbracket \wedge (\sigma, n_2) \in \mathcal{A} \llbracket 21 \rrbracket \wedge n = n_1 + n _2\} \\
       \mathcal{A} \llbracket x \rrbracket  & = & \{ (\sigma, \sigma(x)) \} \\
    \mathcal{A} \llbracket 21 \rrbracket  & = & \{ (\sigma, 21) \} \\
    \mathcal{A} \llbracket x + 21 \rrbracket & = & \{ (\sigma, n) \mid (\sigma, n_1) \in (\sigma, \sigma(x)) \wedge (\sigma, n_2) \in (\sigma, 21) \wedge n = n_1 + n_2 \} \\
    & = & \{ (\sigma, n) \mid n = \sigma(x) + 21 \} \\
    & = & \{ (\sigma, \sigma(x) + 21) \} \\
      \mathcal{C} \llbracket x := x + 21\rrbracket & = & \{ (\sigma, \sigma[x \mapsto \sigma(x) + 21]) \} \\
     \mathcal{C} \llbracket x := x + 21; x := x + 21 \rrbracket & = & \{ (\sigma, \sigma') \mid \exists \sigma''.  ((\sigma, \sigma'') \in  \{ (\sigma, \sigma[x \mapsto \sigma(x) + 21]) \} \wedge (\sigma'', \sigma') \in \mathcal{C} \llbracket x := x + 21\rrbracket )\}\\
      \mathcal{C} \llbracket x := x + 21; x := x + 21 \rrbracket & = & \{ (\sigma, \sigma') \mid \exists \sigma''.  \sigma'' = \sigma[x \mapsto \sigma(x) + 21] \} \wedge (\sigma'', \sigma') \in \mathcal{C} \llbracket x := x + 21\rrbracket )\}\\
       \mathcal{C} \llbracket x := x + 21\rrbracket & = & \{ (\sigma'', \sigma''[x \mapsto n]) \mid (\sigma'', n) \in \mathcal{A} \llbracket x + 21 \rrbracket \} \\
     \mathcal{A} \llbracket x + 21\rrbracket & = & \{ (\sigma'', n) \mid (\sigma'', n_1) \in \mathcal{A} \llbracket x \rrbracket \wedge (\sigma'', n_2) \in \mathcal{A} \llbracket 21 \rrbracket \wedge n = n_1 + n _2\} \\
       \mathcal{A} \llbracket x \rrbracket  & = & \{ (\sigma'', \sigma''(x)) \} \\
         \mathcal{A} \llbracket x \rrbracket  & = & \{ (\sigma'', \sigma(x) + 21) \} \\
    \mathcal{A} \llbracket 21 \rrbracket  & = & \{ (\sigma'', 21) \} \\
    \mathcal{A} \llbracket x + 21 \rrbracket & = & \{ (\sigma'', n) \mid (\sigma'', n_1) \in (\sigma'', \sigma''(x)) \wedge (\sigma'', n_2) \in (\sigma'', 21) \wedge n = n_1 + n_2 \} \\
    & = & \{ (\sigma'', n) \mid n = \sigma(x) + \sigma(x) + 21 \} \\
    & = & \{ (\sigma'', \sigma(x) + 42) \} \\
    \mathcal{C} \llbracket x := x + 21; x := x + 21 \rrbracket & = & \{(\sigma, \sigma[x \mapsto \sigma(x) + 42 ]) \}
    \end{eqnarray*}

{\bf Denotation of}  (x := x + 42):
      \begin{eqnarray*}  \mathcal{C} \llbracket x := x + 42 \rrbracket & = & \{ (\sigma, \sigma[x \mapsto n]) \mid (\sigma, n) \in \mathcal{A} \llbracket x + 42 \rrbracket \} \\
    \mathcal{A} \llbracket x + 42 \rrbracket & = & \{ (\sigma, n) \mid (\sigma, n_1) \in \mathcal{A} \llbracket x \rrbracket \wedge (\sigma, n_2) \in \mathcal{A} \llbracket 42 \rrbracket \wedge n = n_1 + n _2\} \\
    \mathcal{A} \llbracket x \rrbracket  & = & \{ (\sigma, \sigma(x)) \} \\
    \mathcal{A} \llbracket 42 \rrbracket  & = & \{ (\sigma, 42) \} \\
    \mathcal{A} \llbracket x + 42 \rrbracket & = & \{ (\sigma, n) \mid (\sigma, n_1) \in (\sigma, \sigma(x)) \wedge (\sigma, n_2) \in (\sigma, 42) \wedge n = n_1 + n_2 \} \\
    & = & \{ (\sigma, n) \mid n = \sigma(x) + 42 \} \\
    & = & \{ (\sigma, \sigma(x) + 42) \} \\
     \mathcal{C} \llbracket x := x + 42 \rrbracket & = & \{ (\sigma, \sigma[x \mapsto n]) \mid n = \sigma(x) + 42 \} \\
     & = &  \{ (\sigma, \sigma[x \mapsto \sigma(x) + 42 ])\} \\
   \end{eqnarray*}
   
    \item
      To reduce the amount of repeated text, we'll refer to 
      $(x \texttt{:= }1\texttt{; do } x \texttt{:= } x + 1 \texttt{ until } x < 0)$ 
      as $c_a$, and we'll refer to $\texttt{while true do } c$ as $c_b$. 
      
      \begin{eqnarray*}
        \mathcal{C} \llbracket c_b \rrbracket & = & \text{fix}(F) \\
        F(f) & = & \{ (\sigma, \sigma) \mid (\sigma, \texttt{false}) \in 
                                           \mathcal{B} \llbracket b \rrbracket \}\\
            & \cup & \{(\sigma, \sigma'') 
               \mid (\sigma, \texttt{true}) \in \mathcal{B} \llbracket b\rrbracket
             \wedge (\sigma, \sigma') \in \mathcal{c} \llbracket c \rrbracket 
             \wedge (\sigma', \sigma'') \in f \} \\
      \end{eqnarray*}
      Substituting \texttt{true} in for $b$ give the following equation (which is 
      only true for commands with the same form as $c_b$):
      \begin{eqnarray*}
        F'(f) & = & \{ (\sigma, \sigma) \mid (\sigma, \texttt{false}) \in 
                               \mathcal{B} \llbracket \texttt{true} \rrbracket \}\\
            & \cup & \{(\sigma, \sigma'') 
    \mid (\sigma, \texttt{true}) \in \mathcal{B} \llbracket \texttt{true}\rrbracket
             \wedge (\sigma, \sigma') \in \mathcal{C} \llbracket c \rrbracket 
             \wedge (\sigma', \sigma'') \in f \} \\
      \end{eqnarray*} 
      Since $\mathcal{B} \llbracket \texttt{true} \rrbracket = \{(\sigma, 
      \texttt{true})\}$, the pair $(\sigma, \texttt{false})$ will never be a subset 
      of $\mathcal{B} \llbracket \texttt{true} \rrbracket$ and the set preceding 
      the union operator will be the empty set. Likewise, $(\sigma, \texttt{true})$ 
      is always a subset of $\{(\sigma, \texttt{true})\}$, so we can refine our 
      definition of $F'$ as follows:
            \begin{eqnarray*}
        F'(f) & = &  \{(\sigma, \sigma'') 
    					    \mid (\sigma, \sigma') \in \mathcal{C} \llbracket c \rrbracket 
                \wedge (\sigma', \sigma'') \in f \} \\
      \end{eqnarray*} 
      Now we'll show that $f_n$ is undefined on all stores for all $n \in
      \mathbb{N}$ using induction (and the higher order function $F'$):
      \begin{eqnarray*}
        P(n)  & \triangleq & f_n = \emptyset \\
        P(0): & & f_0 = F^0(\emptyset) = \emptyset \text{ } \checkmark \\
        P(n+1): & & f_{n+1} = F(f_n) \\
                & = & \{(\sigma, \sigma'') 
    					    \mid (\sigma, \sigma') \in \mathcal{C} \llbracket c \rrbracket 
                \wedge (\sigma', \sigma'') \in f_n \}
      \end{eqnarray*}
      Using our inductive hypothesis, we can conclude that $f_n = \emptyset$ which
      in turn means that the second rule in the set comprehension (e.g. $(\sigma',
      \sigma'') \in f_n$) is unsatisfiable. So the right hand side of the equation
      is just the empty set, as desired:
      
      $$f_{n+1} = \emptyset \text{ } \checkmark$$ 
      
      Therefore, $P(n)$ hold for all $n \in \mathbb{N}$ and the fixed point of $F'$ 
      is:
      
      $$ \text{fix}(F') = \bigcup_{i \geq 0} F'^{i}(\emptyset) = \emptyset \cup 
      \emptyset \cup ... \cup \emptyset = \emptyset.$$
      
      In short, $c_b$ is guaranteed not to terminate.
      
    \item {\bf Denotation of}  (x := x):
       \begin{eqnarray*}  \mathcal{C} \llbracket x := x \rrbracket & = & \{ (\sigma, \sigma[x \mapsto n]) \mid (\sigma, n) \in \mathcal{A} \llbracket x \rrbracket \} \\
    \mathcal{A} \llbracket x \rrbracket & = & \{ (\sigma, \sigma)\} \\
     \mathcal{C} \llbracket x := x \rrbracket & = & \{ (\sigma, \sigma[x \mapsto \sigma(x)]) \}\\
     & = & \{(\sigma, \sigma)\}\\
   \end{eqnarray*}
   
   
   {\bf Denotation of}   ($\texttt{if} (x = x+1) \hspace{1mm} \texttt{then} \hspace{1mm} x := 0$):
       \begin{eqnarray*}  \mathcal{C} \llbracket \texttt{if} (x = x+1) \hspace{1mm} \texttt{then}\hspace{1mm} x := 0 \rrbracket & = & \{ (\sigma, \sigma) \mid (\sigma, \texttt{false}) \in \mathcal{B} \llbracket x = x + 1 \rrbracket \} \cup \{ (\sigma, \sigma') \mid (\sigma, \texttt{true}) \in \mathcal{B}\\  \llbracket x = x + 1 \rrbracket \wedge (\sigma, \sigma') \in \mathcal{C}\llbracket x := 0 \rrbracket \} \\
    \mathcal{A} \llbracket x \rrbracket & = & \{ (\sigma, \sigma)\} \\
     \mathcal{C} \llbracket x := x \rrbracket & = & \{ (\sigma, \sigma[x \mapsto \sigma(x)]) \}\\
     & = & \{(\sigma, \sigma)\}\\
   \end{eqnarray*}

  \end{enumerate}

% Exercise 3 %%%%%%%%%%%%%%%%%%%%%%%%%%%%%%%%%%%%%%%%%%%%%%%%%%%%%%%%%%%%%%%%%%%%%%%
  \item Our loop invariant is: $r = n - qm \wedge q \geq 0.$

% Exercise 4 %%%%%%%%%%%%%%%%%%%%%%%%%%%%%%%%%%%%%%%%%%%%%%%%%%%%%%%%%%%%%%%%%%%%%%%
  \item
  \begin{enumerate} [(a)]
    \item
    \item
  \end{enumerate}

% Exercise 5 %%%%%%%%%%%%%%%%%%%%%%%%%%%%%%%%%%%%%%%%%%%%%%%%%%%%%%%%%%%%%%%%%%%%%%%
  \item
  \begin{enumerate} [(a)]
    \item
    \item
  \end{enumerate}

\end{enumerate}
\end{document}