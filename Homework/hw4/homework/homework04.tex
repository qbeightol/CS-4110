\newif\ifbeamer\beamerfalse 

\documentclass[11pt]{article}

\usepackage{../tex/jnf}
\usepackage{../tex/angle}

\renewcommand{\labelenumi}{\textbf{(\alph{enumi})}}
\solutionfalse
\begin{document}

\bigredheader{Homework \#4}

\vspace*{-1.25\bigskipamount}

\paragraph{ChangeLog}
\begin{itemize}
\item Version 1 (Wednesday, September 24): Initial release.
\item Version 2 (Friday, September 26): Fixed typo in informal description of 
$\impfnt{do}$-$\impfnt{until}$ loops. A \impfnt{do}-\impfnt{until}
loop $(\until{c}{b})$ executes $c$ one or more times until $b$ becomes
$\TRUE$, not until $b$ becomes $\FALSE$.
\end{itemize}

\paragraph{Due} Wednesday, October 1, 2014 at 11:59pm.

\paragraph{Instructions} This assignment may be completed with one partner. 
You and your partner should submit a single solution on CMS. Please do
not offer or accept any other assistance on this assignment. Late
submissions will not be accepted.

\begin{exercise}
\begin{enumerate}
\item Extend the denotational semantics of IMP to handle the following
  commands:
%
\[
c ::= \dots \mid \IF~b~\THEN~c \mid \until{c}{b}
\]
%
Operationally, these commands behave as follows. A one-armed
conditional $(\IF~b~\THEN~c)$ executes the body $c$ only if $b$
evaluates to $\TRUE$, whereas a \impfnt{do}-\impfnt{until} loop
$(\until{c}{b})$ executes $c$ one or more times until $b$ becomes
$\TRUE$.

\item Extend Hoare logic with rules to handle one-armed conditionals
  and \impfnt{do}-\impfnt{until} loops.
\end{enumerate}
\end{exercise}

\begin{exercise}
  A simple way to prove two programs equivalent is to show that they
  denote the same mathematical object. In particular, this is often
  dramatically simpler than reasoning using the operational
  semantics. Using the denotational semantics, prove the following
  equivalences:

\begin{itemize}
\item $(x := x + 21; x := x + 21) \sim x := x + 42$

\item $(x := 1; \until{x := x + 1}{x \lt 0}) \sim (\while{\TRUE}{c})$

\item $(x := x) \sim (\IF~(x = x + 1)~\THEN~x := 0)$

\end{itemize}
\end{exercise}

\begin{exercise}
Find a suitable invariant for the loop in the following program:
%
\[
\begin{array}{l}
\{ x = n \wedge y = m \}\\[1ex]
\assgn{r}{x};\\
\assgn{q}{0};\\
\WHILE~(y \leq r)~\DO~\{\\
\quad \assgn{r}{r - y};\\
\quad \assgn{q}{q + 1};\\
\}\\[1ex]
\{ r \lt m \wedge n = r + m * q \}\\
\end{array}
\]
Note: you do \emph{not} have to give the proof of this partial
correctness statement in Hoare Logic, but you may wish to complete the
proof to convince yourself your invariant is suitable.

\end{exercise}

\begin{exercise}
  Prove the following partial correctness specifications in Hoare
  Logic. You may write your proofs using a ``decorated program'', in
  which you indicate the assertion at each program point, invariants
  for loops, and any implications needed for the
  \rulename{Consequence} rule.

\begin{center}
\begin{minipage}{.5\textwidth}
\[
\begin{array}{rl}
\text{\textbf{(a)}} & \{ \exists n.~x = 2 \times n + 1 \}\\
&\;\WHILE~y \gt 0~\DO\\
& \qquad \{ \exists n.~x = 2 \times n + 1 \wedge y \gt 0 \}\\
&\qquad x := x + 2\\
& \qquad \{ \exists n.~x = 2 \times n + 1 \wedge y \gt 0 \}\\
& \{ \exists n.~x = 2 \times n + 1 \wedge \neg (y \gt 0) \} \Rightarrow\\
& \{ \exists n.~x = 2 \times n + 1 \wedge y \leq 0) \} \Rightarrow\\
&\{ \exists n.~x = 2 \times n + 1 \}
\end{array}
\]
\vspace*{1.75cm}
\end{minipage}\begin{minipage}{.5\textwidth}
\[
\begin{array}{rl}
\text{\textbf{(b)}} & \{ x = i \wedge y = j \wedge 0 \leq i \} \Rightarrow\\
& \{0 = 0 \wedge x = i \wedge y = j \wedge 0 \leq i \} \\
& \;\ASSGN{z}{0}\\
& \{ z = 0 \wedge x = i \wedge y = j \wedge 0 \leq i\} \Rightarrow\\
& \{ z = (i - x) \times y \wedge x \geq 0 \} \\
& \;\WHILE~(x \gt 0)~\DO~\{\\
&  \{ z = (i - x) \times y \wedge x \geq 0 \wedge x \gt 0\} \Rightarrow\\
&  \{ z = (i - x) \times y \wedge x + 1 \geq 0\} \\
& \qquad\ASSGN{z}{z + y}\\
& \{ z = (i - (x + 1)) \times y \wedge  x + 1 \geq 0 x + 1 \gt 0 \}\\
& \qquad\ASSGN{x}{x - 1}\\
&  \{ z = (i - x) \times y \wedge x \geq 0 \wedge x \gt 0\}\\
& \;\};\\
& \{ z = (i - x) \times j \wedge x \geq 0 \} \\
& \;\ASSGN{y}{0}\\
& \{ things here \}\\
& \{ x = 0 \wedge y = 0 \wedge z = i \times j\} \\
\end{array}
\]
\end{minipage}
\end{center}

\end{exercise}


\begin{exercise}
\begin{enumerate}
\item If we replaced the \rulename{While} rule with the following
  rule,
%
\[
\infrule[While-Alt1]
{ \vdash \hrtrp{P}{c}{(b \Rightarrow P) \wedge (\neg b \Rightarrow Q)} }
{ \vdash \hrtrp{(b \Rightarrow P) \wedge (\neg b \Rightarrow Q) }{\while{b}{c}}{Q} }
{}
\]
%
would the logic still be (relatively) complete? Prove it or give a
counterexample.


\item What if we replaced it with the following rule instead?
%
\[
\infrule[While-Alt2]
{ \vdash \hrtrp{P}{c}{P} }
{ \vdash \hrtrp{P}{\while{b}{c}}{P \wedge \neg b} }
{}
\]
%
Prove it or give a counterexample.

\end{enumerate}
\end{exercise}

\begin{debriefing} \hfill\\[-4ex]
\begin{enumerate*}
\item How many hours did you spend on this assignment? 
\item Would you rate it as easy, moderate, or difficult? 
\item How deeply do you feel you understand the material it covers (0\%–100\%)? 
\item If you have any other comments, we would like to hear them!
  Please write them here or send email to
  \mtt{jnfoster@cs.cornell.edu}.
\end{enumerate*}
\end{debriefing}

\end{document}

