\newif\ifbeamer\beamerfalse

\documentclass[11pt]{article}

\usepackage{../tex/jnf}
\usepackage{../tex/angle}
\usepackage{bussproofs}

\solutionfalse

\renewcommand{\labelenumi}{\textbf{(\alph{enumi})}}

\begin{document}


\vspace*{-1.25\bigskipamount}

\begin{center}\LARGE CS 4110\\ \Large Homework 2\\ 
\vspace{1mm}
\large Ameya Acharya (apa52) and Quinn Beightol (qeb2)\\ \end{center}


\begin{exercise}
Prove the following theorem using the large-step semantics:

\begin{theorem*}
If $\sigma(i)$ is even and $\CONFIG{\sigma}{\while{b}{i := i+2}}
\stepsto \sigma'$ then $\sigma'(i)$ is also even.
\end{theorem*}
\end{exercise}

We prove this by induction on $\CONFIG{\sigma}{\while{b}{i := i+2}} \stepsto \sigma'$.

%Base case: $\sigma(i)$ is even (given). $\checkmark$\\

%Inductive step:\\
Let  $\sigma$ be some arbitrary store. We know that $\sigma(i)$ is even (given); let $\sigma(i) = n$.\\

%Consider the following $\sigma[i \mapsto n]$, where $n = 2x$ for some $x \in \mathbb{Z}$ (since $\sigma(i)$ is even).\\

$\emph{Case 1}$: $b$ is false.

\begin{mathpar}


\inferrule*[right=WHILE-F]{{ \CONFIG{\sigma}{b} \stepsto {false} 
}}{
\CONFIG{\sigma}{\while{b}{i := i+2}} \stepsto \sigma 
}
\end{mathpar}

Since $\CONFIG{\sigma}{\while{b}{i := i+2}} \stepsto \sigma'$, $\sigma' = \sigma$. Therefore, $\sigma'(i) = \sigma(i)$, so $\sigma'(i)$ is even. \checkmark \\
 
$\emph{Case 2}$: $b$ is true.\\

\begin{mathpar}


 { \inferrule *[right=WHILE-T] { { \CONFIG{\sigma}{b} \stepsto {true} 
} {\inferrule *[right=ASSGN] {\inferrule *[right=VAR]{n = \sigma(i)}{\CONFIG{\sigma}{i + 2} \stepsto {n + 2}}}{\CONFIG{\sigma}{i := n+2} \stepsto \sigma'}} {\inferrule {\CONFIG{\sigma'}{\while{b}{i := i+2}} \stepsto \sigma''}{}}}
{\CONFIG{\sigma}{\while{b}{i := i+2}} \stepsto \sigma''}}
\end{mathpar}

By the inductive hypothesis applied to ${\CONFIG{\sigma'}{\while{b}{i := i+2}}} \stepsto \sigma''$:\\

\indent If $b$ is false, then we may apply Case 1, and conclude that $\sigma' = \sigma''$. By the ASSGN rule used above, we know that $\sigma'$ = $\sigma [i \mapsto n + 2]$. Since $n$ is even (given), we know that $n + 2$ is even. \checkmark \\ 

\indent If $b$ is true, then we may recursively apply the above argument. Because we know that\\ ${\CONFIG{\sigma'}{\while{b}{i := i+2}}} \stepsto \sigma''$, we know that evaluation of ${\CONFIG{\sigma'}{\while{b}{i := i+2}}}$ terminates.\\
Therefore, we know that there exists some intermediate $\sigma_n$ such that $$\CONFIG{\sigma_n}{b}\stepsto false.$$

When we evaluate ${\CONFIG{\sigma_n}{\while{b}{i := i+2}}}$, we will apply Case 1. From the induction hypothesis, we know that before we arrived at $\sigma_n$, $i := i + 2$ occurred some $m$ times. Therefore, $\sigma_n$ = $\sigma [i \mapsto n + 2m]$, which is even. \checkmark

This concludes the case, and the proof. $\blacksquare$

%, and conclude that $\sigma' = \sigma''$. By the ASSGN rule used above, we know that $\sigma'$ = $\sigma [i \mapsto n + 2]$. Since $n$ is even (given), we know that $n + 2$ is even. \checkmark \\ 



\begin{exercise}
Recall that IMP commands are equivalent if they always evaluate to the
same result:
\[
c_1 \sim c_2 ~~\defeq~~\forall \sigma,\sigma' \in \Set{Store}.~ <\sigma,c_1> \Downarrow \sigma' \Longleftrightarrow <\sigma,c_2> \Downarrow \sigma'.
\]
%
For each of the following pair of IMP commands, either use the
large-step operational semantics to prove they are equivalent or give
a concrete counter-example showing that they are not equivalent. You
may assume that the language has been extended with operators such as
$x != 0$.

\begin{enumerate}
\item 
$\assign{x}{a};~\assign{y}{a}$
\qquad and \qquad
$\assign{y}{a};~\assign{x}{a}$,\\
where $a$ is an arbitrary arithmetic expression\\

We show that these commands are not equivalent with the following counter-example.

Consider a store $\sigma$ with $x = 3, y = 4$. Let our expression $a$ be $x + y$.\\

Evaluating $ x := a; \hspace{1mm} y := a$:

$\CONFIG{\sigma}{x := x + y; y := x + y}\\
\rightarrow \CONFIG{\sigma}{x := x + y; y := x + y}\\
\rightarrow \CONFIG{\sigma}{x := 3 + y; y := x + y}\\
\rightarrow \CONFIG{\sigma}{x := 3 + 4; y := x + y}\\
\rightarrow \CONFIG{\sigma}{x := 7; y := x + y}\\
\rightarrow \CONFIG{\sigma'}{y := x + y}, \mathrm{where} \hspace{1mm} \sigma' = \sigma[x \mapsto 7]\\
\rightarrow \CONFIG{\sigma'}{y := 7 + y}\\
\rightarrow \CONFIG{\sigma'}{y := 7 + 4}\\
\rightarrow \CONFIG{\sigma'}{y := 11}\\
\rightarrow \sigma'', \mathrm{where} \hspace{1mm} \sigma'' = \sigma [x \mapsto 7, y \mapsto 11]$\\

Evaluating $ y := a; \hspace{1mm} x := a$:

$\CONFIG{\sigma}{y := x + y; x := x + y}\\
\rightarrow \CONFIG{\sigma}{y := 3 + y; x := x + y}\\
\rightarrow \CONFIG{\sigma}{y := 3 + 4; x := x + y}\\
\rightarrow \CONFIG{\sigma}{y := 7; x := x + y}\\
\rightarrow \CONFIG{\sigma'}{x := x + y}, \mathrm{where} \hspace{1mm} \sigma' = \sigma[y \mapsto 7]\\
\rightarrow \CONFIG{\sigma'}{x := 3 + y}\\
\rightarrow \CONFIG{\sigma'}{x := 3 + 7}\\
\rightarrow \CONFIG{\sigma'}{x := 10}\\
\rightarrow \sigma'',\mathrm{where} \hspace{1mm} \sigma'' = \sigma [x \mapsto 10, y \mapsto 7]$\\

We see that these two do $\emph{not}$ produce the same results. Therefore, the above IMP commands are not equivalent.\\

\item 
$\WHILE~b~\DO~c$ 
\qquad and \qquad
$\IF~b~\THEN~(\WHILE~b~\DO~c); c~\ELSE~\SKIP$,\\
where $b$ is an arbitrary boolean expression and $c$ an arbitrary
command.

We show that these commands are not equivalent with the following counter-example.

Consider a store $\sigma$ with $i = 2$. Let $b$ be $i = 2$ and $c$ be $i := i + 2$.\\

Evaluating $\WHILE~i = 2~\DO~i := i + 2$:\\
 
$ \CONFIG{\sigma}{\WHILE~i = 2~\DO~i := i + 2}\\
\stepsone \CONFIG{\sigma}{\WHILE~2 = 2~\DO~i := i + 2}\\
\stepsone \CONFIG{\sigma}{i := i + 2; \WHILE~i = 2~\DO~i := i + 2}\\
\stepsone \CONFIG{\sigma}{i := 2 + 2; \WHILE~i = 2~\DO~i := i + 2}\\
\stepsone \CONFIG{\sigma}{i := 4; \WHILE~i = 2~\DO~i := i + 2}\\
\stepsone \CONFIG{\sigma'}{\WHILE~i = 2~\DO~i := i + 2}, \mathrm{where} \hspace{1mm} \sigma' = \sigma[i \mapsto 4]\\
\stepsone \CONFIG{\sigma'}{\WHILE~4 = 2~\DO~i := i + 2}\\
\stepsone{\sigma', \mathrm{where} \hspace{1mm} \sigma' = \sigma[i \mapsto 4]}$\\

Evaluating $\IF~i  = 2~\THEN~(\WHILE~ i = 2~\DO~ i := i + 2);  i := i + 2~\ELSE~\SKIP$:\\

$ \CONFIG{\sigma}{\IF~i  = 2~\THEN~(\WHILE~ i = 2~\DO~ i := i + 2);  i := i + 2~\ELSE~\SKIP}\\
\stepsone  \CONFIG{\sigma}{\IF~2  = 2~\THEN~(\WHILE~ i = 2~\DO~ i := i + 2);  i := i + 2~\ELSE~\SKIP}\\
\stepsone  \CONFIG{\sigma}{(\WHILE~ i = 2~\DO~ i := i + 2);  i := i + 2}\\
\stepsone  \CONFIG{\sigma}{(i := i + 2; \WHILE~ i = 2~\DO~ i := i + 2);  i := i + 2}\\
\stepsone  \CONFIG{\sigma}{(i := 2 + 2; \WHILE~ i = 2~\DO~ i := i + 2);  i := i + 2}\\
\stepsone  \CONFIG{\sigma}{(i := 4; \WHILE~ i = 2~\DO~ i := i + 2);  i := i + 2}\\
\stepsone  \CONFIG{\sigma'}{(\WHILE~ i = 2~\DO~ i := i + 2);  i := i + 2}, \mathrm{where} \hspace{1mm} \sigma' = \sigma[i \mapsto 4]\\
\stepsone  \CONFIG{\sigma'}{(\WHILE~ 4 = 2~\DO~ i := i + 2);  i := i + 2}\\
\stepsone  \CONFIG{\sigma'}{i := i + 2}\\
\stepsone  \CONFIG{\sigma'}{i := 4 + 2}\\
\stepsone  \CONFIG{\sigma'}{i := 6}\\
\stepsone \sigma'', \mathrm{where} \hspace{1mm} \sigma' = \sigma[i \mapsto 6]\\
$

We see that these two do $\emph{not}$ produce the same results. Therefore, the above IMP commands are not equivalent.\\


\item $\while{x~\mathord{!}\mathord{=}~0}{x := 0}$\qquad and \qquad $x:= 0 * x$\\

%%%%%%%%%%%%%%%%%%%%%%%%%%%%%%%%%%%%%%%%%%%%%%%%%%%%%%%%%%%%%%%%%%%%%%%%%%%%%%%%%%%

% new version of the proof %

To prove that $\while{x~\mathord{!}\mathord{=}~0}{x := 0}$ and $x:= 0 * x$ are equivalent, we'll consider $\sigma(x)$ undefined, $\sigma(x) = 0$ and $\sigma(x) = n$, where $n \neq 0$, and show that either 
\begin{enumerate}
\item $\left< \sigma, \while{x~\mathord{!}\mathord{=}~0}{x := 0} \right> \Downarrow \sigma'$ and $\left< \sigma, x:= 0 * x \right> \Downarrow \sigma'$
\item $\neg(\left< \sigma, \while{x~\mathord{!}\mathord{=}~0}{x := 0} \right> \Downarrow \sigma')$ and $\neg(\left< \sigma, x:= 0 * x \right> \Downarrow \sigma')$ \\*
\end{enumerate}
which is equivalent to proving $\left< \sigma, \while{x~\mathord{!}\mathord{=}~0}{x := 0} \right> \Downarrow \sigma' \Leftrightarrow \left< \sigma, x:= 0 * x \right> \Downarrow \sigma'$\\*


\underline{$\sigma(x)$ undefined:} \\*
In this case, its impossible for either command to step forward--if you attempted to create a derivation showing that either command could be evaluated, you'd eventually reach a case where you need to use the value of $\sigma(x)$ to support another inference. Consequently, any assertions that $\left< \sigma, c \right> \sigma'$ (where $c$ is one of the commands in the prompt), will be false for any choice of $\sigma$, $\sigma'$, and $c$.

\underline{$\sigma(x) = 0$:}\\*
Here, we'll analyze how each command steps forward (Note that each configuration will step to exactly one final store because IMP is deterministic):

\begin{prooftree}
\AxiomC{$\left< \sigma, x \right> \Downarrow 0$}
\AxiomC{$\left< \sigma, 0, \right> \Downarrow 0$}
\AxiomC{$0=0$}
\TrinaryInfC{$\left< \sigma,x~\mathord{!}\mathord{=}~0 \right> \Downarrow \FALSE$}
\UnaryInfC{$\left< \sigma, \while{x~\mathord{!}\mathord{=}~0}{x := 0} \right> \Downarrow \sigma$}
\end{prooftree}

\begin{prooftree}
\AxiomC{$\left< \sigma, 0 \right> \Downarrow 0$}
\AxiomC{$\sigma(x) = 0$}
\UnaryInfC{$\left< \sigma,x \right> \Downarrow 0$}
\AxiomC{$0 = 0 \times 0$}
\TrinaryInfC{$\left<\sigma, 0 * x \right> \Downarrow 0$}
\UnaryInfC{$\left< \sigma, x:= 0 * x \right> \Downarrow \sigma[x \mapsto 0]$}
\end{prooftree}

Notice $\sigma$ already maps $x$ to 0, so $\sigma[x \mapsto 0] = \sigma$. Now let's consider the statements  $\left< \sigma, \while{x~\mathord{!}\mathord{=}~0}{x := 0} \right> \Downarrow \sigma'$ and $\left< \sigma, x:= 0 * x \right> \Downarrow \sigma'$ for arbitrary values of $\sigma'$. 

If $\sigma'$ happens to be equal to $\sigma$ then we can use our previous derivations to conclude that both statements are true simultaneously. Otherwise, both will be false, which concludes this branch of the proof.

\underline{$\sigma(x) = n$ where $n \neq 0$} \\*

\footnotesize 
\begin{prooftree}
\AxiomC{$\sigma(x) = n$}
\UnaryInfC{$\left< \sigma, x, \right> \Downarrow n$}
	\AxiomC{$\left< \sigma, 0 \right> \Downarrow 0$}
		\AxiomC{$n \neq 0$}
\TrinaryInfC{$\left< \sigma, x != 0 \right> \Downarrow \TRUE$} 
	\AxiomC{$\left< \sigma, 0 \right> \Downarrow 0$}
	\UnaryInfC{$\left< \sigma, x:= 0 \right> \Downarrow \sigma[x \mapsto 0]$}
		\AxiomC{$\left< \sigma [x\mapsto 0], x \right> \Downarrow 0$}
		\AxiomC{$\left< \sigma[x \mapsto 0], 0, \right> \Downarrow 0$}
		\AxiomC{$0=0$}
		\TrinaryInfC{$\left< \sigma[x\mapsto 0],x~\mathord{!}\mathord{=}~0 \right> 
					  \Downarrow \FALSE$}
		\UnaryInfC{$\left< \sigma[x\mapsto 0], 
					\while{x~\mathord{!}\mathord{=}~0}{x := 0}
					\right> \Downarrow \sigma[x\mapsto 0]$}
\TrinaryInfC{$\left< \sigma, \while{x~\mathord{!}\mathord{=}~0}{x := 0} \right> \Downarrow \sigma[x \mapsto 0]$}
\end{prooftree}
\normalsize

\begin{prooftree}
\AxiomC{$\left< \sigma, 0 \right> \Downarrow 0$}
\AxiomC{$\sigma(x) = n$}
\UnaryInfC{$\left< \sigma,x \right> \Downarrow n$}
\AxiomC{$0 = 0 \times n$}
\TrinaryInfC{$\left<\sigma, 0 * x \right> \Downarrow 0$}
\UnaryInfC{$\left< \sigma, x:= 0 * x \right> \Downarrow \sigma[x \mapsto 0]$}
\end{prooftree}

Again, the statements $\left< \sigma, \while{x~\mathord{!}\mathord{=}~0}{x := 0} \right> \Downarrow \sigma'$ and $\left< \sigma, x:= 0 * x \right> \Downarrow \sigma'$ for arbitrary values of $\sigma'$ will either be true simultaneously or false simultaneously. $\blacksquare$


\newpage
% old version %
OLD VERSION

To prove that $\while{x~\mathord{!}\mathord{=}~0}{x := 0}$ and $x:= 0 * x$ are equivalent, we'll first assume that $\left< \sigma, \while{x~\mathord{!}\mathord{=}~0}{x := 0} \right> \Downarrow \sigma'$ and then show that $\left< \sigma, x:=0*x \right>$ necessarily steps to the same $\sigma'$. Then we'll show that if $\left< \sigma, x:= 0 * x \right> \Downarrow \sigma'$ (and note that this is a different $\sigma'$ from the $\sigma'$ in the last sentence), then $\left< \sigma, \while{x~\mathord{!}\mathord{=}~0}{x := 0} \right> \Downarrow \sigma'$:

$\Rightarrow$

Assume that $\left<\sigma, \while{x~\mathord{!}\mathord{=}~0}{x := 0}\right> \Downarrow \sigma'$. There are two possible derivation trees for this assumption:

\underline{While-F}:

\begin{prooftree}
\AxiomC{$\left< \sigma, x \right> \Downarrow 0$}
\AxiomC{$\left< \sigma, 0, \right> \Downarrow 0$}
\AxiomC{$0=0$}
\TrinaryInfC{$\left< \sigma,x~\mathord{!}\mathord{=}~0 \right> \Downarrow \FALSE$}
\UnaryInfC{$\left< \sigma, \while{x~\mathord{!}\mathord{=}~0}{x := 0} \right> \Downarrow \sigma$}
\end{prooftree}

so we can conclude $\sigma' = \sigma$ and $\sigma(x) = 0$

Now let's evaluate $\left< \sigma, x:= 0 * x\right>$:

\begin{prooftree}
\AxiomC{$\left< \sigma, 0 \right> \Downarrow 0$}
\AxiomC{$\sigma(x) = 0$}
\UnaryInfC{$\left< \sigma,x \right> \Downarrow 0$}
\AxiomC{$0 = 0 \times 0$}
\TrinaryInfC{$\left<\sigma, 0 * x \right> \Downarrow 0$}
\UnaryInfC{$\left< \sigma, x:= 0 * x \right> \Downarrow \sigma[x \mapsto 0]$}
\end{prooftree}

since $\sigma$ already maps $x$ to 0, $\sigma[x \mapsto 0] = \sigma$, and, in this case, the commands produce the same final store given the same starting store.

\underline{While-T}
\footnotesize 
\begin{prooftree}
\AxiomC{$\sigma(x) = n$}
\UnaryInfC{$\left< \sigma, x, \right> \Downarrow n$}
	\AxiomC{$\left< \sigma, 0 \right> \Downarrow 0$}
		\AxiomC{$n \neq 0$}
\TrinaryInfC{$\left< \sigma, x != 0 \right> \Downarrow \TRUE$} 
	\AxiomC{$\left< \sigma, 0 \right> \Downarrow 0$}
	\UnaryInfC{$\left< \sigma, x:= 0 \right> \Downarrow \sigma[x \mapsto 0]$}
		\AxiomC{$\left< \sigma [x\mapsto 0], x \right> \Downarrow 0$}
		\AxiomC{$\left< \sigma[x \mapsto 0], 0, \right> \Downarrow 0$}
		\AxiomC{$0=0$}
		\TrinaryInfC{$\left< \sigma[x\mapsto 0],x~\mathord{!}\mathord{=}~0 \right> 
					  \Downarrow \FALSE$}
		\UnaryInfC{$\left< \sigma[x\mapsto 0], 
					\while{x~\mathord{!}\mathord{=}~0}{x := 0}
					\right> \Downarrow \sigma[x\mapsto 0]$}
\TrinaryInfC{$\left< \sigma, \while{x~\mathord{!}\mathord{=}~0}{x := 0} \right> \Downarrow \sigma[x \mapsto 0]$}
\end{prooftree}
\normalsize

which tells us that $\sigma(x) = n$, where $n \neq 0$. Again, let's consider the evaluation of the second command:

\begin{prooftree}
\AxiomC{$\left< \sigma, 0 \right> \Downarrow 0$}
\AxiomC{$\sigma(x) = n$}
\UnaryInfC{$\left< \sigma,x \right> \Downarrow n$}
\AxiomC{$0 = 0 \times n$}
\TrinaryInfC{$\left<\sigma, 0 * x \right> \Downarrow 0$}
\UnaryInfC{$\left< \sigma, x:= 0 * x \right> \Downarrow \sigma[x \mapsto 0]$}
\end{prooftree}

So in both cases, the commands step to the same ending store given the same starting store. I.e.

$$\left< \sigma, \while{x~\mathord{!}\mathord{=}~0}{x := 0} \right> \Downarrow \sigma' \Rightarrow \left< \sigma, x:= 0 * x \right> \Downarrow \sigma'$$


$\Leftarrow$
Here, we'll assume $\left < \sigma, x:=0 * x \right> \Downarrow \sigma'$, and analyze the only possible derivation for this assumption:

\begin{prooftree}
\AxiomC{$\left< \sigma, 0 \right> \Downarrow 0$}
\AxiomC{$\sigma(x) = n$}
\UnaryInfC{$\left< \sigma,x \right> \Downarrow n$}
\AxiomC{$0 = 0 \times n$}
\TrinaryInfC{$\left<\sigma, 0 * x \right> \Downarrow 0$}
\UnaryInfC{$\left< \sigma, x:= 0 * x \right> \Downarrow \sigma[x \mapsto 0]$}
\end{prooftree}

We know very little about $\sigma$ (in fact, the only thing we know is that it maps $x$ to some value $n$, so for the next step in the proof we'll consider two possible values of $n$.

\underline{$n = 0$:}\\*

\begin{prooftree}
\AxiomC{$\left< \sigma, 0 \right> \Downarrow 0$}
\AxiomC{$\sigma(x) = 0$}
\UnaryInfC{$\left< \sigma,x \right> \Downarrow 0$}
\AxiomC{$0 = 0 \times 0$}
\TrinaryInfC{$\left<\sigma, 0 * x \right> \Downarrow 0$}
\UnaryInfC{$\left< \sigma, x:= 0 * x \right> \Downarrow \sigma[x \mapsto 0]$}
\end{prooftree}

Since $\sigma$ already mapped $x$ to 0, $\sigma = \sigma[x \mapsto 0]$. So both commands stepped to the same final store when given the same starting store.

\underline{$n \neq 0$:}\\*

\footnotesize 
\begin{prooftree}
\AxiomC{$\sigma(x) = n$}
\UnaryInfC{$\left< \sigma, x, \right> \Downarrow n$}
	\AxiomC{$\left< \sigma, 0 \right> \Downarrow 0$}
		\AxiomC{$n \neq 0$}
\TrinaryInfC{$\left< \sigma, x != 0 \right> \Downarrow \TRUE$} 
	\AxiomC{$\left< \sigma, 0 \right> \Downarrow 0$}
	\UnaryInfC{$\left< \sigma, x:= 0 \right> \Downarrow \sigma[x \mapsto 0]$}
		\AxiomC{$\left< \sigma [x\mapsto 0], x \right> \Downarrow 0$}
		\AxiomC{$\left< \sigma[x \mapsto 0], 0, \right> \Downarrow 0$}
		\AxiomC{$0=0$}
		\TrinaryInfC{$\left< \sigma[x\mapsto 0],x~\mathord{!}\mathord{=}~0 \right> 
					  \Downarrow \FALSE$}
		\UnaryInfC{$\left< \sigma[x\mapsto 0], 
					\while{x~\mathord{!}\mathord{=}~0}{x := 0}
					\right> \Downarrow \sigma[x\mapsto 0]$}
\TrinaryInfC{$\left< \sigma, \while{x~\mathord{!}\mathord{=}~0}{x := 0} \right> \Downarrow \sigma[x \mapsto 0]$}
\end{prooftree}
\normalsize

Again, both commands stepped to the same final store given the same starting store.
$\blacksquare$



%%%%%%%%%%%%%%%%%%%%%%%%%%%%%%%%%%%%%%%%%%%%%%%%%%%%%%%%%%%%%%%%%%%%%%%%%%%%%%%%%%




\end{enumerate}
\end{exercise}

\begin{exercise}
Let $\CONFIG{\sigma}{c} \stepsone \CONFIG{\sigma'}{c'}$ be the
small-step operational semantics relation for IMP. Consider the
following definition of the multi-step relation:

\begin{mathpar}
\inferrule*[right=R1]{ 
}{
\CONFIG{\sigma}{c} \stepsone\kleenestar \CONFIG{\sigma}{c}
}

\inferrule*[right=R2]{
\CONFIG{\sigma}{c} \stepsone \CONFIG{\sigma'}{c'} \qquad
\CONFIG{\sigma'}{c'} \stepsone\kleenestar \CONFIG{\sigma''}{c''}
}{ \CONFIG{\sigma}{c} \stepsone\kleenestar \CONFIG{\sigma''}{c''} 
}
\end{mathpar}

\noindent Prove the following theorem, which states that
$\stepsone\kleenestar$ is transitive.

\begin{theorem*} If \( \CONFIG{\sigma}{c} \stepsone\kleenestar
\CONFIG{\sigma'}{c'} \) and \( \CONFIG{\sigma'}{c'}
\stepsone\kleenestar \CONFIG{\sigma''}{c''} \) then \(
\CONFIG{\sigma}{c} \stepsone\kleenestar \CONFIG{\sigma''}{c''} \).
\end{theorem*}

Because of our axiom R1 and our inference rule R2, we have an exhaustive set of cases against which we may pattern-match the expression \( \CONFIG{\sigma}{c} \stepsone\kleenestar
\CONFIG{\sigma'}{c'} \).\\

We complete the following proof by induction on  \( \CONFIG{\sigma}{c} \stepsone\kleenestar
\CONFIG{\sigma'}{c'} \).\\


{\it Case R1}:
% We know \( \CONFIG{\sigma}{c} \stepsone\kleenestar \CONFIG{\sigma'}{c'} \) and \( \CONFIG{\sigma'}{c'} \stepsone\kleenestar \CONFIG{\sigma''}{c''} \).\\ 
By R1, \( \CONFIG{\sigma}{c} \stepsone\kleenestar \CONFIG{\sigma}{c} \), so $\sigma = \sigma'$ and $c = c'$.\\
By substituting $\sigma$ and $c$ for $\sigma'$ and $c'$, we get  \(
\CONFIG{\sigma}{c} \stepsone\kleenestar \CONFIG{\sigma''}{c''} \). \checkmark \\

{\it Case R2}: 
%We know that \CONFIG{\sigma}{c} \stepsone\kleenestar \CONFIG{\sigma'}{c'} and \CONFIG{\sigma'}{c'} \stepsone\kleenestar \CONFIG{\sigma''}{c''}.
We know that \( \CONFIG{\sigma}{c} \stepsone\kleenestar \CONFIG{\sigma'}{c'} \), and we may break it down into the following, for some $\sigma_n$ and $c_n$:

\centerline{\( \CONFIG{\sigma}{c} \stepsone \CONFIG{\sigma_n}{c_n} \)}
\centerline{ \( \CONFIG{\sigma_n}{c_n} \stepsone\kleenestar \CONFIG{\sigma'}{c'} \).}

\vspace{3mm}

%Therefore, we may apply the inductive hypothesis to  \( \CONFIG{\sigma}{c} \stepsone\kleenestar \CONFIG{\sigma_n}{c_n} \) and \( \CONFIG{\sigma_n}{c_n} \stepsone\kleenestar \CONFIG{\sigma'}{c'} \).

Therefore, we may apply the inductive hypothesis to \( \CONFIG{\sigma_n}{c_n} \stepsone\kleenestar \CONFIG{\sigma'}{c'} \) and  \( \CONFIG{\sigma'}{c'} \stepsone\kleenestar \CONFIG{\sigma''}{c''} \) to conclude that \( \CONFIG{\sigma_n}{c_n} \stepsone\kleenestar \CONFIG{\sigma''}{c''} \).\\

By applying R1 to \( \CONFIG{\sigma}{c} \stepsone \CONFIG{\sigma_n}{c_n} \), we know that \( \CONFIG{\sigma}{c} \stepsone\kleenestar \CONFIG{\sigma_n}{c_n} \).\\

Therefore, we may apply the inductive hypothesis to  \( \CONFIG{\sigma}{c} \stepsone\kleenestar \CONFIG{\sigma_n}{c_n} \) and \( \CONFIG{\sigma_n}{c_n} \stepsone\kleenestar \CONFIG{\sigma''}{c''} \) to conclude that  \( \CONFIG{\sigma}{c} \stepsone\kleenestar \CONFIG{\sigma''}{c''} \). \checkmark \\

This concludes the case and the proof. $\blacksquare$


\end{exercise}

\begin{exercise}

\newcommand{\THROW}[1]{\ensuremath{\impfnt{throw}~#1}}
\newcommand{\TRYCATCH}[3]{\ensuremath{\impfnt{try}~#1~\impfnt{with}~#2~\DO~#3}}

In this exercise, you will extend the IMP language with
exceptions. These exceptions are intended to behave like the analogous
constructs found in languages such as Java. We will proceed in several
steps.

First, we fix a set of exceptions, which will ranged over by
metavariables $e$, and we extend the syntax of the language with new
commands for throwing and handling exceptions:
%
\[
\begin{array}{r@{~}c@{~}l}
 c & ::= & \SKIP \\
& \mid & x := a\\
& \mid & c_1 ; c_2\\
& \mid & \cond{b}{c_1}{c_2} \\
& \mid & \while{b}{c} \\
& \mid & \shade{\THROW{e}} \\
& \mid & \shade{\TRYCATCH{c_1}{e}{c_2}}
\end{array}
\]
%
Intuitively, evaluating a command either yields a modified store or a
pair comprising a modified store and an (uncaught) exception. We let
metavariables $r$ range over such results:
%
\[
r ::= \sigma \mid (\sigma,e)
\]
%

Second, we change the type of the large-step evaluation relation so it
yields a result instead of a store: $<\sigma,c> \stepsto r$.

Third, we will extend the large-step semantics rules so they handle
\impfnt{throw} and \impfnt{try} commands. This is your task in this
exercise. 

Informally, $\THROW~e$ should return exception $e$, and
$\TRYCATCH{c_1}{e}{c_2}$ should execute $c_1$ and return the result it
produces, unless the result contains an exception $e$, in which case
it should discard $e$ executes the handler $c_2$. You will also need
to modify many other rules so they have the right type and also
propagate exceptions.

\begin{mathpar}

 { \inferrule { { \CONFIG{\sigma}{b} \stepsto {true}} \hspace{2mm} {\CONFIG{\sigma}{c_1} \stepsto \CONFIG{\sigma'}{e}}} {\CONFIG{\sigma}{ \IF~b~\THEN~c_1~\ELSE~c_2} \stepsto \CONFIG{\sigma'}{e}}}
 
  { \inferrule { { \CONFIG{\sigma}{b} \stepsto {false}} \hspace{2mm} {\CONFIG{\sigma}{c_2} \stepsto \CONFIG{\sigma'}{e}}} {\CONFIG{\sigma}{ \IF~b~\THEN~c_1~\ELSE~c_2} \stepsto \CONFIG{\sigma'}{e}}}
  
%While-F safe.\\

  { \inferrule { { \CONFIG{\sigma}{b} \stepsto {true}} \hspace{2mm} {\CONFIG{\sigma}{c} \stepsto \CONFIG{\sigma'}{e}}} {\CONFIG{\sigma}{\WHILE~b~\DO~c} \stepsto \CONFIG{\sigma'}{e}}}
  
    { \inferrule { { \CONFIG{\sigma}{b} \stepsto {true}} \hspace{2mm} {\CONFIG{\sigma}{c} \stepsto \sigma'} {\CONFIG{\sigma'}{\WHILE~b~\DO~c} \stepsto \CONFIG{\sigma''}{e}}} {\CONFIG{\sigma}{\WHILE~b~\DO~c} \stepsto \CONFIG{\sigma''}{e}}}
    
%    $\THROW~e$ should return exception $e$
     { \inferrule {\\\\}{\CONFIG{\sigma}{\THROW~e} \stepsto \CONFIG{\sigma}{e}}}
     
    
%$\TRYCATCH{c_1}{e}{c_2}$ should execute $c_1$ and return the result it
%produces, unless the result contains an exception $e$, in which case
%it should discard $e$ executes the handler $c_2$

%{\inferrule{\\\\}{\}

      { \inferrule {\CONFIG{\sigma}{c_1} \stepsto v} {\CONFIG {\sigma}{{\TRYCATCH{c_1}{e}{c_2}}} \stepsto v}}
      
      { \inferrule {\CONFIG{\sigma}{c_1} \stepsto \CONFIG{\sigma}{e_1}} {\CONFIG {\sigma}{{\TRYCATCH{c_1}{e}{c_2}}} \stepsto  \CONFIG{\sigma}{e_1}}}
      
        { \inferrule {\CONFIG{\sigma}{c_1} \stepsto \CONFIG{\sigma}{e} \\ {\CONFIG{\sigma}{c_2} \stepsto v'}} {\CONFIG {\sigma}{{\TRYCATCH{c_1}{e}{c_2}}} \stepsto v'}}
        
           { \inferrule {\CONFIG{\sigma}{c_1} \stepsto \CONFIG{\sigma}{e} \\ {\CONFIG{\sigma}{c_2} \stepsto \CONFIG{\sigma}{e_n}}} {\CONFIG {\sigma}{{\TRYCATCH{c_1}{e}{c_2}}} \stepsto \CONFIG{\sigma}{e_n}}}



  
 

 
%{\inferrule {\inferrule *[right=VAR]{n = \sigma(i)}{\CONFIG{\sigma}{i + 2} \stepsto {n + 2}}}{\CONFIG{\sigma}{i := n+2} \stepsto \sigma'}} {\inferrule {\CONFIG{\sigma'}{\while{b}{i := i+2}} \stepsto \sigma''}{}}}
%{\CONFIG{\sigma}{\while{b}{i := i+2}} \stepsto \sigma''}}
\end{mathpar}

\end{exercise}

\begin{debriefing} \hfill\\[-4ex]
\begin{enumerate*}
\item How many hours did you spend on this assignment? 
\item Would you rate it as easy, moderate, or difficult? 
\item How deeply do you feel you understand the material it covers (0\%–100\%)? 
\item If you have any other comments, we would like to hear them!
  Please write them here or send email to
  \mtt{jnfoster@cs.cornell.edu}.
\end{enumerate*}
\end{debriefing}
\end{document}

