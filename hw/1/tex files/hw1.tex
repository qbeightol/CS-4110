\documentclass[10pt, oneside]{article}
\usepackage{amssymb, amsmath, geometry, graphicx, bussproofs, enumerate}
\geometry{letterpaper}


\begin{document}
\noindent Quinn Beightol (qeb2) and Ameya Acharya (apa52) \\*
\noindent CS 4110 \\*
\noindent HW 1 \\*

\begin{enumerate} [1.]

\item 
	\begin{enumerate}[(a)]
	\item $ $
	
	\begin{prooftree}
		\AxiomC{$2 = 1 + 1$}
		\RightLabel{(Add)}
		\UnaryInfC{
		$\left< \sigma, \texttt{1+1} \right> 
		 \rightarrow 
		 \left< \sigma, \texttt{2} \right>$}
		\RightLabel{(Assgn1)}
		\UnaryInfC{
		$\left< \sigma, \texttt{x:=1+1; (x ?: x:=1+x; 0)} \right>
		 \rightarrow
		 \left< \sigma, \texttt{x:=2; (x ?: x:=1+x; 0)} \right>$}
	\end{prooftree}
	
	\item $ $
	
	$$\begin{array}{cl}
	  & \left< \sigma, \texttt{x:=2; (x ?: x:=1+x;0)} \right>  \\
	 \rightarrow & \left< \sigma', \texttt{(x ?: x:=1+x; 0)} \right> \text{, where } \sigma' = \sigma[x \mapsto 2]  \\
	  \rightarrow & \left< \sigma', \texttt{(x ?: x:= 1 + 2; 0)} \right> \\
	  \rightarrow & \left< \sigma', \texttt{(x ?: x:=3; 0)} \right> \\
	  \rightarrow & \left< \sigma'', \texttt{x ?: 0}\right> \text{, where } \sigma'' = \sigma'[x \mapsto 3] \\
	  \rightarrow & \left< \sigma'', \texttt{3 ?: 0} \right> \\
	  \rightarrow & \left< \sigma'', \texttt{3} \right>
	\end{array}$$
	
	\item new rules:
	
	\begin{prooftree}
		\AxiomC{
		$\left< \sigma, e_1 \right>
		 \rightarrow
		 \left< \sigma', e'_1 \right>$}
		\RightLabel{Elvis1}
		\UnaryInfC{
		$\left< \sigma, e_1 \texttt{?:} e_2 \right>
		 \rightarrow
		 \left< \sigma', e'_1 \texttt{?:} e_2 \right>$}
	\end{prooftree}
	
	\begin{prooftree}
		\AxiomC{
		$\left< \sigma, e_2 \right>
		 \rightarrow
		 \left< \sigma', e'_2 \right>$}
		\RightLabel{Elvis2}
		\UnaryInfC{
		$\left< \sigma, n \texttt{?:} e_2 \right>
		 \rightarrow
		 \left< \sigma', n \texttt{?:} e'_2 \right>$} 
	\end{prooftree}
	
	Elvis3 and Elvis4 retain their old definitions.
			
	\end{enumerate}
	

\item 
	\begin{enumerate}[(a)]
	\item No, $\left< \sigma, \texttt{(5+2)/(3+2)} \right>$ could step to $\left< \sigma, \texttt{7/(3+2)} \right>$ or $\left< \sigma, \texttt{(5+2)/5} \right>$.
	\item Yes.
	\item No, there aren't any rules that allow $\left< \sigma, \texttt{7/0} \right>$ to progress any further, yet \texttt{7/0} is not an integer.
	\end{enumerate}
	
\item
	\begin{enumerate}[(a)]
	\item note: I'm going to call the set of strings, "\textbf{Str.}" to help differentiate a string, $s$, in the set of strings, from the set \textbf{Str.}
	
	\begin{prooftree}
		\AxiomC{}
		\UnaryInfC{$\epsilon \in \textbf{Str.}$}
	\end{prooftree}
	
	\begin{prooftree}
		\AxiomC{$s \in \textbf{Str.}$}
		\UnaryInfC{$c::s \in \textbf{Str.}$}
	\end{prooftree}
	
	\item We'd need to prove that the base case ($P(\epsilon)$) holds, and that if $P(s)$ holds for some $s \in \textbf{Str.}$, then $P(c::s)$ for some $c$ in the fixed alphabet. 
	
	\item $ $
	\begin{prooftree}
		\AxiomC{$(s,n) \in length$}
		\UnaryInfC{$(c::s, n+1) \in length$}
	\end{prooftree}
	
	\end{enumerate}
	
\item To answer this question, I think its helpful to consider a particular example--specifically, $n = 6$. Clearly, 6 can't be expressed as $2^i$ for some $i \in \mathbb{N}$, so what went wrong? The proof is correct in arguing that $n$ can be expressed as $2 \times m$ (it's, true by definition of "evenness"), and its also correct in arguing that the inductive hypothesis holds for $m$ (because $m = 3$, and therefore the preposition--which can roughly be restated as $\text{even}(n) \implies \exists i \in \mathbb{N} \text{ such that } n = 2 ^ i$--is also true, albeit vacuously. What goes wrong is asserting that $m$ can be expressed as $2^j$ simply because the inductive hypothesis held. The inductive hypothesis only says that $m=2^j$ if $m$ is even. And in this case $m$ is odd, so the inductive hypothesis can't be used to conclude $m=2^j$. 
\end{enumerate}


\end{document}