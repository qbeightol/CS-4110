\documentclass[10pt, oneside]{article}

\usepackage{amssymb, amsmath, bussproofs, enumerate, fancyhdr, geometry, listings, stmaryrd}
\geometry{letterpaper}
\pagestyle{fancy}

\lhead{Ameya Acharya (apa52) and Quinn Beightol (qeb2)}
\rhead{CS4110: Homework 9}


\begin{document}

\begin{enumerate}[1.]
% Exercise 1 %%%%%%%%%%%%%%%%%%%%%%%%%%%%%%%%%%%%%%%%%%%%%%%%%%%%%%%%%%%%%%%%%%%%%%
	\item
	\begin{lstlisting}[language = ML]
let fib : int -> int = 
  let f : (int -> int) ref ref = ref (ref (fun (n: int) -> 42)) in 
  f := ref (fun (n: int) -> 
              let (r: int) = 
                if n = 0 then 1
                else if n = 1 then 1
                else !(!f) (n - 1) + !(!f) (n - 2) in
              let (old_fib: int -> int) = !(!f) in
              f := ref (fun (x: int) -> if x = n then r else old_fib x);
              r
  );
  !(!f);
	\end{lstlisting}

% Exercise 2 %%%%%%%%%%%%%%%%%%%%%%%%%%%%%%%%%%%%%%%%%%%%%%%%%%%%%%%%%%%%%%%%%%%%%%
	\item
	\begin{enumerate}[(a)]
		\item 
		$$\phi \vee \psi \equiv \neg (\neg \phi \wedge \neg \psi) \equiv \forall P. ([\phi \rightarrow P] \wedge [\psi \rightarrow P]) \rightarrow P$$
%					Idea: we could encode bottom as $\forall \alpha. \alpha$ (the idea being
%					that, like $\bot$, this type cannot be inhabited), which in turn 
%					allows us to define $\neg \phi$ as $\phi \rightarrow \bot = \phi
%					\rightarrow \forall \alpha. \alpha$.
%					
%					Using that encoding we could write: 
%					$$\phi \vee \psi \equiv \neg (\neg \phi \wedge \neg \psi)
%						\equiv ([\phi \rightarrow \forall \alpha. \alpha] \times [\psi \rightarrow \forall \alpha. \alpha]) \rightarrow \forall \alpha. \alpha$$
%						
%					Another Idea: maybe we could use unit instead of bottom?
		\item
%		\begin{eqnarray*}
%			\mathcal{T} \llbracket \textsf{unit} \rrbracket & = & \textsf{unit} \\
%			\mathcal{T} \llbracket \tau_1 \rightarrow \tau_2 \rrbracket & = & \mathcal{T} \llbracket \tau_1 \rrbracket \rightarrow \mathcal{T} \llbracket \tau_2 \rrbracket \\
%			\mathcal{T} \llbracket \tau_1 + \tau_2 \rrbracket & = & ([\mathcal{T} \llbracket \tau_1 \rrbracket \rightarrow \forall \alpha. \alpha] \times [\mathcal{T} \llbracket \tau_2 \rrbracket \rightarrow \forall \alpha. \alpha]) \rightarrow \forall \alpha. \alpha \\
%		\end{eqnarray*}
		
				\begin{eqnarray*}
			\mathcal{T} \llbracket \textsf{unit} \rrbracket & = & \textsf{unit} \\
			\mathcal{T} \llbracket \tau_1 \rightarrow \tau_2 \rrbracket & = & \mathcal{T} \llbracket \tau_1 \rrbracket \rightarrow \mathcal{T} \llbracket \tau_2 \rrbracket \\
			\mathcal{T} \llbracket \tau_1 + \tau_2 \rrbracket & = & \forall \alpha. ([\mathcal{T}\llbracket \tau_1 \rrbracket \rightarrow \alpha] \times [\mathcal{T} \llbracket \tau_2 \rrbracket \rightarrow \alpha]) \rightarrow \alpha \\
		\end{eqnarray*}
		
		\item
		\begin{eqnarray*}
			\mathcal{E} \llbracket \textsf{inl}_{\tau_1 + \tau_2} \rrbracket & = & \Lambda \alpha. \lambda e_1:\mathcal{T} \llbracket \tau_1 \rrbracket.\lambda p: (\mathcal{T}\llbracket \tau_1 \rrbracket \rightarrow \alpha) \times (\mathcal{T} \llbracket \tau_2 \rrbracket \rightarrow \alpha). (\#1 \text{ } p) \text{ } e_1\\
			\mathcal{E} \llbracket \textsf{inr}_{\tau_1 + \tau_2} \rrbracket & = & \Lambda \alpha. \lambda e_2:\mathcal{T} \llbracket \tau_2 \rrbracket.\lambda p: (\mathcal{T}\llbracket \tau_1 \rrbracket \rightarrow \alpha) \times (\mathcal{T} \llbracket \tau_2 \rrbracket \rightarrow \alpha). (\#2 \text{ } p) \text{ } e_2\\
			\mathcal{E} \llbracket \textsf{case } e_0 \textsf{ of } e_1 \mid e_2 \rrbracket & = & e_0 [\tau_{out}] \text{ } (\lambda x. e_1 \text{ } x, \lambda y. e_2 \text{ } y) \text{ where $\tau_{out}$ is the translated return type of $e_1$ and $e_2$}\\
		\end{eqnarray*}
	\end{enumerate}

% Exercise 3 %%%%%%%%%%%%%%%%%%%%%%%%%%%%%%%%%%%%%%%%%%%%%%%%%%%%%%%%%%%%%%%%%%%%%%
	\item
	\begin{enumerate}[(a)]
		\item
		$$E ::= ... \mid \texttt{throw } E \mid \texttt{try }\{E\} \texttt{ catch } (C x) \{e\} $$
		
		
		%\begin{prooftree}
		%\AxiomC{}
		%\UnaryInfC{\text{try \{throw new D v\} catch (C x) \{e\} \rightarrow e \{new D v / x \}} E}
		%\end{prooftree}
		
		\begin{prooftree}
		\AxiomC{$ D \leq C $}
		\UnaryInfC{$\texttt{try } \{ \texttt{throw new } D \text{ }  v \} \texttt{catch} (C \text{ } x) \{e\} \rightarrow \{\texttt{(new } D \text{ } v) / x \} $}
	\end{prooftree}
	
	\begin{prooftree}
		\AxiomC{$ D \nleq C $}
		\UnaryInfC{$\texttt{try } \{ \texttt{throw new } D \text{ }  v \} \texttt{catch} (C \text{ } x) \{e\} \rightarrow \texttt{throw new } D  \text{ } v$}
	\end{prooftree}
	
	\begin{prooftree}
		\AxiomC{}
		\UnaryInfC{$\texttt{try } \{ v \} \texttt{ catch} (C \text{ } x) \{e\} \rightarrow  v$}
	\end{prooftree}


		\item
		\item
	\end{enumerate}
\end{enumerate}


























\end{document}