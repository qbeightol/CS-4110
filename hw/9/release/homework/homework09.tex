\newif\ifbeamer\beamerfalse
\documentclass[10pt]{article}

\usepackage{../tex/jnf}
\usepackage{alltt}

\renewcommand{\labelenumi}{\textbf{(\alph{enumi})}}

\begin{document}

\bigredheader{Homework \#9}

\paragraph{ChangeLog}
\begin{itemize}
\item Version 1 (Wednesday, November 19): Initial release.
\item Version 2 (Saturday, November 22): Fixed typo in description of languages in Exercise 2: $\lamfnt{void}$ changed to $()$ and $\lamfnt{null}$ removed.
\end{itemize}


\paragraph{Due} Tuesday, November 25, 2014 by 11:59pm 

\paragraph{Instructions} 
This assignment may be completed with one partner.  You and your
partner should submit a single solution on CMS. Please do not offer or
accept any other assistance on this assignment. Late submissions will
not be accepted.

\begin{exercise}
Naive implementations of recursive functions often redundantly compute
the same results many times. For example, to evaluate
$\mathit{fib}~5$, the value $\mathit{fib}~2$ is computed $3$ times!

One idea for making recursive functions more efficient is to cache and
reuse previously-computed results. This technique is known as
\emph{memoization}. To memoize $\mathit{fib}$ we can build a function
that takes an argument $n$ and checks if $\mathit{fib}~n$ is in the
cache. If so, it returns the result immediately. Otherwise it computes
the result, stores it in the cache, and returns it. As an example,
here is a memoized version of Fibonacci in OCaml that uses an
hashtable for the cache:
%
\begin{alltt}
    let cache : (int,int) Hashtbl.t = Hashtbl.create 43
    let rec fib n =
      try Hashtbl.find cache n with Not_found -> 
        let r = 
          if n = 0 then 1
          else if n = 1 then 1 
          else fib (n-1) + fib (n-2) in
        Hashtbl.add cache n r; 
        r  
\end{alltt}

\noindent In this exercise you will build a memoized version of your
implementation of $\mathit{fib}$ using references. Because we do not
have hashtables, you will have to implement the cache using
references: the first time a result is computed, you should overwrite
the reference at the top-level with one containing a new function that
returns that result if invoked on the same argument in the future and
otherwise computes $\mathit{fib}~n$.
 
We have provided some code to help you get started. Note that there
are two levels of references---one to encode recursion and one to help
you encode the cache.

\bigskip

\(
\begin{array}{l}
\lamfnt{let}~\mathit{fib}~\ty \typ{int} \arrow \typ{int} = \\
\quad \lamfnt{let}~f~\ty~((\typ{int} \arrow \typ{int})~\typ{ref})~\typ{ref} = \lamfnt{ref}~(\lamfnt{ref}~(\lam{n\ty\typ{int}}{42}))~\lamfnt{in}\\
\quad f := \lamfnt{ref}~(\lambda n\ty\typ{int}.\\
\hspace*{25mm} \lamfnt{let}~r \ty \typ{int} = \\
\hspace*{25mm} \quad \lamfnt{if}~n = 0~\lamfnt{then}~1\\ 
\hspace*{25mm} \quad \lamfnt{else}~\lamfnt{if}~n = 1~\lamfnt{then}~1\\ 
\hspace*{25mm} \quad \lamfnt{else}~!(!f)~(n-1)~+~!(!f)~(n-2)~\lamfnt{in}\\[.5em]
\hspace*{25mm} \quad \textit{\fbox{(* Fill in missing code here *)}}\\[.5em]
\quad );\\
\quad !(!f)
\end{array}
\)
\end{exercise}

\begin{exercise} 

In this exercise, we will show that sum types $\tau + \sigma$
can be encoded using product and universal types in System F. The
propositions-as-type principle will help in constructing the
appropriate translations. In particular, we will use one of De
Morgan's laws: $\phi \vee \psi \equiv \neg (\neg \phi \wedge \neg \psi)$.

As the source language, we will use the simply-typed
$\lambda$-calculus extended with sums and unit:
%
\[
\begin{array}{r@{~~}c@{~~}l}
e    &::=& () \mid x \mid e_1\,e_2 \mid \lam{x:\tau}{e} \mid
           \lamfnt{inl}_{\tau_1+\tau_2} e \mid \lamfnt{inr}_{\tau_1+\tau_2} e \mid \lamfnt{case}~e_0~\lamfnt{of}e_1~\mid~{e_2}\\[1ex]
\tau &::=& \lamfnt{unit} \mid \tau_1 \arrow \tau_2 \mid \tau_1 + \tau_2\\
\end{array}
\]
%
As the target language, we will use System F extended with products
and unit (but not sums!):
%
\[
\begin{array}{r@{~~}c@{~~}l}
e    &::=& () \mid x \mid e_1\,e_2 \mid \lam{x:\tau}{e} \mid
           (e_1, e_2) \mid \lamfnt{\#1}\,e \mid \lamfnt{\#2}\,e \mid 
           \Lambda \alpha.~e \mid e~[\tau] \\
\tau &::=& \lamfnt{unit} \mid \tau_1 \arrow \tau_2 \mid \tau_1 \times \tau_2 \mid \alpha \mid \forall \alpha . \tau\\
\end{array}
\]

\begin{enumerate}
\item Give a formula that is equivalent to $\phi \vee \psi$, but which
  only contains logical operators for which there are corresponding
  types in the fragment of System F above.  (Hint: use a universal
  type to encode ``negation''.)

\item Define a translation $\Tr{T}{\cdot}$ on types that takes a type
  in the fragment of simply-typed $\lambda$-calculus listed above and
  produces a type in the fragment of System F extended with products. 

\item Define the corresponding translation $\Tr{E}{\cdot}$ on
  expressions in the fragment of simply-typed $\lambda$ calculus
  listed above. You only need to give the cases for
  $\lamfnt{inl}_{\tau_1+\tau_2}$, $\lamfnt{inr}_{\tau_1+\tau_2}$, and
  $\lamfnt{case}~{e_0}~\lamfnt{of}~{e_1}~\mid~{e_2}$. 

  You may find it helpful to think about types.  In particular, your
  translations should be type preserving in the sense that they should
  map well-typed terms to well-typed terms. However, you do not need
  to prove this property.

\end{enumerate}
\end{exercise}

\begin{exercise}
Suppose we extend Featherweight Java with support for simple
exceptions:
\[
\begin{array}{rrl@{\qquad}l}
e        &   ::=& x \mid e.f \mid e.m(\ol{e}) \kw{new}~C(\ol{e}) \mid (C)~e \mid \shade{\kw{throw}~e} \mid \shade{\kw{try}~\{$e$\}~\kw{catch}~($C$~$x$)~\{~$e$~\}}
\end{array}
\]
%
The typing rules for these new expressions are as follows:
%
\begin{center}
\infrule[T-Throw]
{ \Gamma \vdash e : C \qquad C \subty \kw{Exception} }
{ \Gamma \vdash \kw{throw}~e : D }
{ }
\hfil
\infrule[T-Try]
{ \Gamma \vdash e_1 : D \qquad \Gamma,x : C \vdash e_2 : D }
{ \Gamma \vdash \kw{try}~e_1~\kw{catch}(C~x)~\{ e_2 \} : D }
{ }
\end{center}
%
\begin{enumerate}
\item Extend the definition of evaluation contexts $E$ and the
small-step operational semantics to propagate exceptions, using
expressions of the form \kw{throw}~\kw{new}~C(\ol{v}) to represent
exceptions that have been thrown. You only need to give the new
evaluation contexts and rules.

\item Extend the operational semantics so that down casts step throw a
\texttt{ClassCastException} instead of getting stuck. (You may assume
that the program $P$ already contains suitable definitions of classes
\texttt{Exception} and \texttt{ClassCastException}.)

\item State the progress theorem. (\textbf{Karma:} prove it!)

\end{enumerate}
\end{exercise}

\begin{debriefing} \hfill\\[-4ex]
\begin{enumerate*}
\item How many hours did you spend on this assignment? 
\item Would you rate it as easy, moderate, or difficult? 
\item Did everyone in your study group participate? 
\item How deeply do you feel you understand the material it covers (0\%--100\%)? 
\item If you have any other comments, we would like to hear them!
  Please send email to \texttt{jnfoster@cs.cornell.edu}.
\end{enumerate*}
\end{debriefing}

\end{document}

