\newif\ifbeamer\beamerfalse
\documentclass[11pt]{article}

\usepackage{../tex/jnf}
\usepackage{../tex/angle}

\renewcommand{\labelenumi}{\textbf{(\alph{enumi})}}
\solutiontrue
\begin{document}

\bigredheader{Homework \#5}

\vspace*{-1.25\bigskipamount}

\paragraph{Due} 
%
Wednesday, October 22, 2014 at 11:59pm.

\paragraph{Instructions} 
%
This assignment may be completed with one partner. 
You and your partner should submit a single solution on CMS. Please do
not offer or accept any other assistance on this assignment. Late
submissions will not be accepted.

\paragraph{Exercise}
%
In this assignment, you will implement an automatic verification tool
for the IMP language. Your tool will take annotated programs
containing a precondition, a postcondition, and extra assertions at
certain program points described below, and produce a set of
verification conditions---i.e., assertions that if valid, imply that
the partial correctness specification is valid. These assertions will
(often) be automatically discharged by the Z3 theorem prover.

In more detail, your tool will take programs of the form
$\texttt{@}P\texttt{@}~c~\texttt{@}Q\texttt{@}$ where $P$ and $Q$ are
assertions and $c$ an annotated command, as specified by the following
grammar:
\[
\begin{array}{rr@{\;}l}
s & ::=& \SKIP \mid \ASSGN{x}{a} \mid \impfnt{print}~a \mid \impfnt{test}~b\\[.5em]
c & ::=& s \mid c_1; s_2 \mid c_1;~\texttt{@} P \texttt{@}~c_2\\
& \mid & \IF~b~\THEN~c_1~\ELSE~c_2 \mid \WHILE~b~\DO~\texttt{@}P\texttt{@}~c,
\end{array}\]
%
Note that this grammar requires every loop to be annotated with an
invariant, and every sequential composition to be annotated with an
assertion, except when the second command is ``simple''---i.e., a
skip, assignment, print, or test. Working with these annotated
commands makes it fairly easy to calculate verification conditions
(unlike, say, the weakest liberal preconditions we saw in lecture).

As an example, given the following program
\begin{progeg}
@ x = $m and y = $n @
 t := x; 
 x := y;
 y := t
@ y = $m and x = $n @
\end{progeg}
%
your tool would generate an assertion equivalent to the following,
%
\begin{progeg}
(((x = $m) and (y = $n)) implies ((x = $m) and (y = $n)))
\end{progeg}
%
which could then be discharged using a theorem prover.

More specifically, your tasks are as follows:
\begin{enumerate}
\item Implement a tool that generates verification conditions for
  annotated programs. Please submit your \texttt{vcgen.ml} file.
\item Develop a suite of $10$ or more distinct tests, each in a
  separate file. Your tests should be verified successfully when
  run. We will give extra credit for any tests that expose a bug in
  our implementation. Please submit a single archive
  \texttt{tests.tar.gz} containing the tests in files
  \texttt{test06.imp} through \texttt{test16.imp}.
\end{enumerate}

To help you get started, we have provided starter code on CMS. The
only files you should need to edit submit are \texttt{vcgen.ml} and
your tests.

\paragraph{Notes}
%
\begin{itemize}
\item The \texttt{README.txt} file gives an overview to the code and
  includes instructions on how to build and run the executable. Start
  by reading this file.
\item To distinguish assertions from command blocks, we write
  ``$\texttt{@} P \texttt{@}$'' to indicate an assertion instead of
  ``$\texttt{\{} P \texttt{\}}$''.
\item To distinguish logical variables from program variables in
  assertions, we prefix logical variables with a dollar sign, writing
  ``$\texttt{\$}x$'' for logical variables and ``$x$'' for program
  variables.
\item We represent the boolean expressions \texttt{bexp} that appear
  in programs and assertions \texttt{assn} using different types. To
  convert a boolean expression into an assertion, you may use the
  \texttt{assn\_of\_bexp} we have supplied. Similarly, to convert an
  ordinary arithmetic expression into a logical expression, which can
  be used in an assertion, use the supplied \texttt{lexp\_of\_aexp}
  function. Lastly, to substitute an arithmetic expression for a
  program variable in an assertion, use the supplied
  \texttt{substAssn} function.
\item To represent the verification conditions use an OCaml list
  containing values of type \texttt{assn}. You do not need to worry
  about duplicate elements.

\item To test your code, you will need a working version of the Z3
  theorem prover. Pre-installed binaries for many platforms are
  available at \url{http://z3.codeplex.com/}. After successfully
  installing Z3, you should see the following if you invoke
  \texttt{z3} from the command-line:
\begin{alltt}
[nate@viva:~> z3
Error: input file was not specified.
For usage information: z3 -h
\end{alltt}
\end{itemize}

Good luck!

\begin{debriefing} \hfill\\[-4ex]
\begin{enumerate*}
\item How many hours did you spend on this assignment? 
\item Would you rate it as easy, moderate, or difficult? 
\item Did everyone in your study group participate? 
\item How deeply do you feel you understand the material it covers (0\%--100\%)? 
\item If you have any other comments, we would like to hear them!
  Please write them here or send email to
  \mtt{jnfoster@cs.cornell.edu}.
\end{enumerate*}
\end{debriefing}

\end{document}

