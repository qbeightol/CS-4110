\newif\ifbeamer\beamerfalse

\documentclass[11pt]{article}

\usepackage{../tex/jnf}
\usepackage{../tex/angle}

\solutionfalse

\renewcommand{\labelenumi}{\textbf{(\alph{enumi})}}

\begin{document}

\bigredheader{Homework \#2}

\vspace*{-1.25\bigskipamount}

\paragraph{Due.} 
%
Wednesday, September 17, 2014 at 11:59pm.

\paragraph{Instructions.} 
%
This assignment may be completed with one partner. You and your
partner should submit a single solution on CMS. Please do not offer or
accept any other assistance on this assignment. Late submissions will
not be accepted.

\begin{exercise}
Prove the following theorem using the large-step semantics:

\begin{theorem*}
If $\sigma(i)$ is even and $\CONFIG{\sigma}{\while{b}{i := i+2}}
\stepsto \sigma'$ then $\sigma'(i)$ is also even.
\end{theorem*}
\end{exercise}

\begin{exercise}
Recall that IMP commands are equivalent if they always evaluate to the
same result:
\[
c_1 \sim c_2 ~~\defeq~~\forall \sigma,\sigma' \in \Set{Store}.~ <\sigma,c_1> \Downarrow \sigma' \Longleftrightarrow <\sigma,c_2> \Downarrow \sigma'.
\]
%
For each of the following pair of IMP commands, either use the
large-step operational semantics to prove they are equivalent or give
a concrete counter-example showing that they are not equivalent. You
may assume that the language has been extended with operators such as
$x != 0$.

\begin{enumerate}
\item 
$\assign{x}{a};~\assign{y}{a}$
\qquad and \qquad
$\assign{y}{a};~\assign{x}{a}$,\\
where $a$ is an arbitrary arithmetic expression

\item 
$\WHILE~b~\DO~c$ 
\qquad and \qquad
$\IF~b~\THEN~(\WHILE~b~\DO~c); c~\ELSE~\SKIP$,\\
where $b$ is an arbitrary boolean expression and $c$ an arbitrary
command.

\item $\while{x~\mathord{!}\mathord{=}~0}{x := 0}$\qquad and \qquad $x:= 0 * x$

\end{enumerate}
\end{exercise}

\begin{exercise}
Let $\CONFIG{\sigma}{c} \stepsone \CONFIG{\sigma'}{c'}$ be the
small-step operational semantics relation for IMP. Consider the
following definition of the multi-step relation:

\begin{mathpar}
\inferrule*[right=R1]{ 
}{
\CONFIG{\sigma}{c} \stepsone\kleenestar \CONFIG{\sigma}{c}
}

\inferrule*[right=R2]{
\CONFIG{\sigma}{c} \stepsone \CONFIG{\sigma'}{c'} \qquad
\CONFIG{\sigma'}{c'} \stepsone\kleenestar \CONFIG{\sigma''}{c''}
}{ \CONFIG{\sigma}{c} \stepsone\kleenestar \CONFIG{\sigma''}{c''} 
}
\end{mathpar}

\noindent Prove the following theorem, which states that
$\stepsone\kleenestar$ is transitive.

\begin{theorem*} If \( \CONFIG{\sigma}{c} \stepsone\kleenestar
\CONFIG{\sigma'}{c'} \) and \( \CONFIG{\sigma'}{c'}
\stepsone\kleenestar \CONFIG{\sigma''}{c''} \) then \(
\CONFIG{\sigma}{c} \stepsone\kleenestar \CONFIG{\sigma''}{c''} \).
\end{theorem*}
\end{exercise}

\begin{exercise}

\newcommand{\THROW}[1]{\ensuremath{\impfnt{throw}~#1}}
\newcommand{\TRYCATCH}[3]{\ensuremath{\impfnt{try}~#1~\impfnt{with}~#2~\DO~#3}}

In this exercise, you will extend the IMP language with
exceptions. These exceptions are intended to behave like the analogous
constructs found in languages such as Java. We will proceed in several
steps.

First, we fix a set of exceptions, which will ranged over by
metavariables $e$, and we extend the syntax of the language with new
commands for throwing and handling exceptions:
%
\[
\begin{array}{r@{~}c@{~}l}
 c & ::= & \SKIP \\
& \mid & x := a\\
& \mid & c_1 ; c_2\\
& \mid & \cond{b}{c_1}{c_2} \\
& \mid & \while{b}{c} \\
& \mid & \shade{\THROW{e}} \\
& \mid & \shade{\TRYCATCH{c_1}{e}{c_2}}
\end{array}
\]
%
Intuitively, evaluating a command either yields a modified store or a
pair comprising a modified store and an (uncaught) exception. We let
metavariables $r$ range over such results:
%
\[
r ::= \sigma \mid (\sigma,e)
\]
%

Second, we change the type of the large-step evaluation relation so it
yields a result instead of a store: $<\sigma,c> \stepsto r$.

Third, we will extend the large-step semantics rules so they handle
\impfnt{throw} and \impfnt{try} commands. This is your task in this
exercise. 

Informally, $\THROW~e$ should return exception $e$, and
$\TRYCATCH{c_1}{e}{c_2}$ should execute $c_1$ and return the result it
produces, unless the result contains an exception $e$, in which case
it should discard $e$ executes the handler $c_2$. You will also need
to modify many other rules so they have the right type and also
propagate exceptions.

\end{exercise}

\begin{debriefing} \hfill\\[-4ex]
\begin{enumerate*}
\item How many hours did you spend on this assignment? 
\item Would you rate it as easy, moderate, or difficult? 
\item How deeply do you feel you understand the material it covers (0\%–100\%)? 
\item If you have any other comments, we would like to hear them!
  Please write them here or send email to
  \mtt{jnfoster@cs.cornell.edu}.
\end{enumerate*}
\end{debriefing}
\end{document}

