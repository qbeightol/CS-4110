\newif\ifbeamer\beamerfalse
\documentclass[10pt]{article}

\usepackage{../tex/jnf}
\usepackage{alltt, enumerate, bussproofs}

\renewcommand{\labelenumi}{\textbf{(\alph{enumi})}}

\begin{document}

Ameya Acharya (apa52) and Quinn Beightol (qeb2)\\

Homework 8\\


\begin{exercise}
\begin{enumerate}
\item   Progress does not hold. Progress states that if $\vdash e: \tau$ then e is either a value or $\exists e'. e \rightarrow e'.$ Consider ($\lambda x: unit \text{ } x) + 5$. This is not a value. So, we consider if it can step. We see that the expression is well-typed, because$\lambda x: unit \text{ } x$ has type $int$ according to our new rule. However, there is no value to which this steps. Therefore, our new rule breaks progress. \\

Preservation holds. Preservation states that if $\vdash e: \tau$ and $e \rightarrow e'$ then $\vdash e': \tau$. We are changing the typing rule for a value, which doesn't step. If we consider the case where $\lambda x: unit \text{ } x) + 5$ is the result of a function application, we could use the previous typing rules to show that preservation holds.

\item  Progress holds: we still have all the old evaluation rules and we haven't changed the type rules. Therefore, we can just re-use the old evaluation rules if necessary.\\

Preservation does not hold: consider $(4 + 5) : int$. Using the new rules, we have $(4 + 5) \rightarrow ()$, and $() : unit.$ Under preservation, if $4 + 5$ steps, it is expected to step to an expression of type $int,$ which we do not have. Therefore, preservation does not hold.
 
\end{enumerate}
\end{exercise}

\begin{exercise}
%  The ``erasure'' of a System F term into a pure $\lambda$-calculus
  %term can be defined as follows:
%
%\[
%\begin{array}{rcl}
%\mathit{erase}(x) & = & x\\
%\mathit{erase}(\lam{x \ty \tau}{e}) & = & \lam{x}{\mathit{erase}(e)}\\
%\mathit{erase}(e_1~e_2) & = & (\mathit{erase}(e_1)~ \mathit{erase}(e_2))\\
%\mathit{erase}(\Lam{X}{e}) & = & \lam{z}{\mathit{erase}(e)}~\qquad \text{where}~z~\text{fresh}\\
%\mathit{erase}(e~[\tau]) & =& \mathit{erase}(e)~\lam{z}{z}
%\end{array}
%\]
%
We prove that if $e \arrow e'$ then
$\mathit{erase}(e) \Rightarrow \mathit{erase}(e')$ by induction on the derivation of e.\\

First, we prove the following lemma: $\mathit{erase}(e\{v/x\}) = \mathit{erase}(e)\{\mathit{erase}(v)/x\}$.\\
We proceed by structural induction on $e.$\\

{\sc case} $e = y$:
\begin{itemize}
\item $\mathit{erase}(y \{v/x\}) = \mathit{erase}(y) = y.$
\item $\mathit{erase}(y) \{\mathit{erase}(v) /x\} = (y)\{\mathit{erase} (v)/x\} = y.$\\

\checkmark
\end{itemize}

{\sc case} $e = \lambda y: \tau. e_{body}$:
\begin{itemize}
\item $\mathit{erase}(\lambda y: \tau. e_{body} \{v/x\}) = \mathit{erase}(\lambda y: \tau. e_{body})\{v/x\}$ 
= $\lambda y. \mathit{erase}(e_{body} \{v/x\}).$\\
\item $\mathit{erase}(\lambda y: \tau. e_{body}) \{\mathit{erase}(v)/x\} = (\lambda y: \tau. \mathit{erase} (e_{body})) \{\mathit{erase}(v)/x\} = \lambda y. \mathit{erase} (e_{body}) \{\mathit{erase} (v)/x\}.$\\

\end{itemize}

By the inductive hypothesis, $\mathit{erase}(e_{body}\{v/x\}) = \mathit{erase}(e_{body})\{\mathit{erase}(v)/x\}$. Therefore, our two expressions above are equal, which concludes this case. \checkmark\\

{\sc case} $e = e_1e_2$:
\begin{itemize}
\item $\mathit{erase}((e_1e_2)\{v/x\}) = \mathit{erase}(e_1\{v/x\} \text{ } e_2\{v/x\}) =  \mathit{erase}(e_1\{v/x\})\text{ } \mathit{erase}(e_2\{v/x\})$
\item $\mathit{erase}(e_1e_2)\{\mathit{erase}(v)/x\} =  [\mathit{erase}(e_1)\text{ } \mathit{erase}(e_2)] \{\mathit{erase}(v)/x\} = \mathit{erase}(e_1)\{v/x\} \text{ } \mathit{erase}(e_2)\{\mathit{erase}(v)/x\}$\\
\end{itemize}

By the inductive hypothesis, $\mathit{erase}(e_1\{v/x\}) = \mathit{erase}(e_1)\{\mathit{erase}(v)/x\}$. Therefore, our two expressions above are equal, which concludes this case. \checkmark\\

{\sc case} $e = \Lambda X.e_{body}$: 
\begin{itemize}
\item $\mathit{erase}(\Lambda X.e_{body}\{v/x\}) = \lambda z. \mathit{erase}(e_{body}\{v/x\}) $
%= (\lambda z. \mathit{erase}(e_{body})) \{v/x\}$
\item $\mathit{erase}(\Lambda X.e_{body})\{\mathit{erase}(v)/x\} = (\lambda z. \mathit{erase}(e_{body}))\{\mathit{erase}(v)/x\}$ 
%= (\lambda z. \mathit{erase}(e_{body})) \{v/x\}$\\

\end{itemize}

By the inductive hypothesis, $\mathit{erase}(e_{body}\{v/x\}) = \mathit{erase}(e_{body})\{\mathit{erase}(v)/x\}$. Therefore, our two expressions above are equal, which concludes this case. \checkmark\\


{\sc case} $e = e_{poly}[\tau] $:\\

TODO: what is the rule we use for the first case?\\

\begin{itemize}
\item $\mathit{erase}((e_{poly}[\tau])\{v/x\}) = \mathit{erase}(e_{poly}\{v/x\}[\tau]) = \mathit{erase}(e_{poly} \{v/x\}) \lambda z.z$
\item $(\mathit{erase}(e_{poly}[\tau]))\{\mathit{erase}(v)/x\} = (\mathit{erase}(e_{poly}[\tau]) \lambda z.z)\{\mathit{erase}(v)/x\} = (\mathit{erase}(e_{poly}[\tau])\{v/x\} (\lambda z.z) \{\mathit{erase}(v)/x\}$\\
\end{itemize}

By the inductive hypothesis, $\mathit{erase}(e_{poly}[\tau]\{v/x\}) = \mathit{erase}(e_{poly}[\tau])\{\mathit{erase}(v)/x\}$. Therefore, our two expressions above are equal, which concludes this case and the proof. \checkmark\\

Additionally, we'll prove another lemma: $erase(e\{\tau/X\}) = erase(e)$ \\

\noindent \underline{Proof: by structural induction on $e$}
\begin{enumerate}[\hspace{20pt}]
	\item
	\underline{Case $e = n$:}\\
	$$\text{erase}(e\{\tau/ X\}) = \text{erase}(n\{\tau / X\}) = \text{erase}(n) 	\checkmark$$ \\
	
	\underline{Case $e = x$:}\\
	$$\text{erase}(e\{\tau/ X\}) = \text{erase}(x\{\tau / X\}) = \text{erase}(x)
	\checkmark$$ \\
	
	\underline{Case $e = \lambda x: \tau_{in}.e_{body}$:}\\
	Here, we'll consider $\text{erase}(e\{\tau / X\})$ regardless of whether $
	\tau_{in} = X$. Then we'll show that the lemma holds for lambda terms.
	
	\begin{enumerate}[\hspace{20pt}]
	\item
	\underline{Subcase $\tau_{in} = X$} \\
	$$\text{erase}(e\{\tau / X\}) = \text{erase}(\lambda x:\tau.e_{body}\{\tau/X\}) 
	= \lambda x.\text{erase}(e_{body}\{\tau/X\})$$ \\
	
	\underline{Subcase $\tau_{in} \neq X$} \\
	$$\text{erase}(e\{\tau / X\}) = \text{erase}(\lambda x:\tau_{in}.e_{body}\{\tau/X	\}) 
	= \lambda x.\text{erase}(e_{body}\{\tau/X\})$$ \\
	\end{enumerate}
	
	Now we'll consider $\text{erase}(e)$:
	
	$$\text{erase}(e) = \text{erase}(\lambda x:\tau_{in}.e_{body}) = \lambda x.
	\text{erase}(e_{body})$$
	
	By the inductive hypothesis, $\text{erase}(e_{body} \{\tau/X\}) = \text{erase}	
	(e_{body})$, so $\text{erase}(e \{\tau/X\}) = \text{erase}(e)$, as desired.
	\checkmark \\
	
	\underline{Subcase $e = e_1 \text{ } e_2$}\\
	\begin{eqnarray*}
		& & \text{erase}(e\{\tau / X\} \\
		& = & \text{erase}((e_1 \text{ } e_2) \{\tau / X \})\\
		& = & \text{erase}(e_1\{\tau / X\} \text{ } e_2\{\tau / X\}) \\
		& = & \text{erase}(e_1\{\tau / X\}) \text{ } \text{erase}(e_2\{\tau / X\}) \\
	\end{eqnarray*}
	
	\begin{eqnarray*}
		& & \text{erase(e)} \\
		& = & \text{erase}(e_1 \text{ } e_2) \\
		& = & \text{erase}(e_1) \text{ } \text{erase}(e_2) \\
	\end{eqnarray*}
	
	By the inductive hypothesis, $\text{erase}(e_1\{\tau / X\}) = \text{erase}(e_1)$ 
	and $\text{erase}(e_2\{\tau / X\}) = \text{erase}(e_2)$ so 
	$\text{erase}(e\{\tau / X\}) = \text{erase}(e)$ \checkmark \\
	
	\underline{Case $e = \Lambda Y.e_{body}$:} \\
	
	\begin{enumerate}[\hspace{20pt}]
		\item 
		\underline{Subcase $X = Y$:} \\
		$$\text{erase}(e\{\tau / X\}) = \text{erase}(e = \Lambda \tau.e_{body}\{\tau / 
		X\}) = \lambda z. (\text{erase}(e_{body}\{\tau / X\})$$
		
		\underline{Subcase $X \neq Y$:} \\
		$$\text{erase}(e\{\tau / X\}) = \text{erase}(e = \Lambda Y.e_{body}\{\tau / 
		X\}) = \lambda z. (\text{erase}(e_{body}\{\tau / X\})$$
	\end{enumerate}
	
	Regardless of whether $X = Y$, $\text{erase}(e\{\tau / X\}) = \lambda z. 
	(\text{erase}(e_{body}\{\tau / X\})$.
	
	$$\text{erase}(\Lambda Y.e_{body}) = \lambda z.\text{erase}(e_{body})$$
	
	By the inductive hypothesis, $\text{erase}(e_{body}\{\tau / X\}) = \text{erase}
	(e_{body})$, so the lemma holds in this case as well. \checkmark \\
	
	\underline{Case $e = e_{poly} [\tau]$:} \\
	
	$\text{erase}(e\{\tau / X\}) = \text{erase}((e_{poly} [\tau']) \{\tau / X \})$
	
	\begin{enumerate}[\hspace{20pt}]
		\item
		\underline{Subcase $\tau' = X$:}
		$$\text{erase}(e\{\tau / X\}) = \text{erase}(e_{poly}\{\tau / X\} [\tau]) = 
		\text{erase}(e_{poly}\{\tau / X\}) (\lambda x.x)$$
		
		\underline{Subcase $\tau' \neq X$:} \\
		$$\text{erase}(e\{\tau / X\}) = \text{erase}(e_{poly}\{\tau / X\} [\tau']) = 
		\text{erase}(e_{poly}\{\tau / X\}) (\lambda x.x)$$
	\end{enumerate}
	
	Again, $\text{erase}(e\{\tau / X\})$ is the same in both subcases.
	
	$$\text{erase}(e) = \text{erase}(e_{poly} [\tau']) = \text{erase}(e_{poly}) 
	(\lambda x.x)$$
	
	Using the inductive hypothesis, we can conclude that $\text{erase}(e_{poly}) = 
	\text{erase}(e_{poly}\{\tau / X\})$. So the lemma holds in this case as well.
	\checkmark \\
	
\end{enumerate}

By the structural induction principle, $erase(e\{\tau/X\}) = erase(e)$ for all 
$e$, $\tau$, and $X$. \\



Now, we return to our original proof: if $e \arrow e'$ then
$\mathit{erase}(e) \Rightarrow \mathit{erase}(e')$.\\

{\sc case} $\beta$ {\sc -reduction}:\\
\begin{itemize}

\item $e = (\lambda x: \tau. e_{body}) v$

\item $e' = e_{body}\{v/x\}$

\item $\mathit{erase}(e) = \mathit{erase}((\lambda x:\tau. e_{body}) v) = \mathit{erase}(\lambda x.\mathit{erase}(e_{body})) \mathit{erase}(v) = \lambda x.\mathit{erase}(e_{body}) v \rightarrow \mathit{erase}(e_{body})\{v/x\}$

\item $\mathit{erase}(e') = \mathit{erase}(e_{body} \{v/x\}) = \mathit{erase}(e_{body})\{v/x\}$\\

\end{itemize}

So, $\mathit{erase}(e) \Rightarrow \mathit{erase}(e'). \checkmark$ \\

{\sc case type-reduction}:

\begin{itemize}
\item $e = \Lambda X. e_{body} [\tau]$
\item $e' = e_{body} \{\tau / x\}$
\item $\mathit{erase}(e) = \mathit{erase}(\Lambda X. e_{body}[\tau]) = \mathit{erase}(\Lambda X. e_{body}) \lambda z.z = \lambda z.\mathit{erase}(e_{body}) \lambda z. z \rightarrow \mathit{erase}(e_{body})$
\item $\mathit{erase}(e') = \mathit{erase}(e_{body} \{\tau/X\})$ TODO complete this case
\end{itemize}


{\sc case context}: 
\begin{itemize}
\item $e = E[e_1]$
\item $e' = E[e_1']$
\item $e \rightarrow e'$ (assume)
\item $\mathit{erase}(e) = \mathit{erase}(E[e_1])$
\item $\mathit{erase}(e') = \mathit{erase}(E[e_1'])$
\end{itemize}

By {\sc context}, $e_1 \rightarrow e_1'$. So, by the inductive hypothesis, $\mathit{erase}(e_1) \Rightarrow \mathit{erase}(e_1')$. Therefore, we may conclude that $\mathit{erase}(E[e_1]) \Rightarrow \mathit{erase}(E[e_1']). \texttt{ } \checkmark$ 


  

%(where
%to avoid any confusion we let ``$\arrow$'' stand for the evaluation
%relation in System F and ``$\Rightarrow$'' stand for the evaluation
%relation for the pure $\lambda$-calculus). Although this property is
%``obvious,'' proving it rigorously is still good practice!
\end{exercise}

\begin{exercise}
\begin{enumerate}
\begin{minipage}{.5\textwidth}
\item  $\Lambda A. \lambda x:A. x$   \\
\item $\Lambda A. \lambda x:A. \lambda y:A. x$\\
$\Lambda A. \lambda x:A. \lambda y:A. y$
\item $\Lambda A. \Lambda B. \lambda x:A. \lambda f:A \rightarrow B. \text{ } f x$ \\
\item $\Lambda A. \Lambda B. \lambda x:A. \lambda y:B. \text{inl } x$\\
$\Lambda A. \Lambda B. \lambda x:A. \lambda y:B. \text{inr } y$
\end{minipage}\begin{minipage}{.5\textwidth}
\item Nothing\\
\item $\Lambda A. \lambda x:A. \lambda b:\text{bool}. x $\\
\item  $\Lambda A. \lambda x:A. \lambda y:A. \lambda b:\text{bool}. x $\\
$\Lambda A. \lambda x:A. \lambda y:A. \lambda b:\text{bool}. y $\\
$\Lambda A. \lambda x:A. \lambda y:A. \lambda b:\text{bool}. \text{if } b \text{ then } x \text{ else } y $\\
$\Lambda A. \lambda x:A. \lambda y:A. \lambda b:\text{bool}. \text{if } b \text{ then } y \text{ else } x$\\
\bigskip\bigskip
\end{minipage}
\end{enumerate}
\end{exercise}

\begin{exercise}
  For each of the following, write ``Yes'' if it is okay to allow
  these types to be in the subtype relation or ``No'' if not. In
  addition, if your answer is ``No'' give a counterexample that shows
  how type soundness would break.

\begin{itemize}
\item $\typ{int} \subty \typ{unit}$: 
			Yes \\
\item $\{ l : \top \} \subty \{ l : \typ{bool} \}$: 
			No \\
			Consider 
			
			$$(\lambda r:\{l:\typ{bool}\}.r.l) \text{ } \{l = 5\}$$ 
			
			We can use our typing rules to conclude that this
			expression has type $\typ{bool}$, but when we use our dynamic semantics to 
			evaluate the expression, we'll get an $\typ{int}$, which isn't a subtype of 
			$\typ{bool}$.\\ 

\item $\{ \} \subty \{ x : \top \}$: 
			No \\
			Consider 
			
			$$(\lambda r:\{x:\top\}.r.x) \text{ } \{\}$$ 
			
			Again, we can give this expression a type (specifically, $\top$), but 
			when we evaluate the expression, we get stuck (the input record has no "x"
			field) even though the expression isn't a value.\\
			
\item $(\top \times \{ x \ty \typ{unit} \}) \subty (\{\} \times \top)$:
			Yes \\
			
			
\item $(\{ x \ty \typ{int} \} \arrow \typ{int}) \subty (\{ x \ty \typ{int}, y \ty 			\typ{int} \} \arrow \typ{int})$: 
			Yes \\
			
\item $(\{ x \ty \typ{int}, y \ty \typ{int} \} \arrow \typ{int}) \subty (\{ x \ty 
			\typ{int} \} \arrow \typ{int})$:
			No\\
			Consider 
			
			$$(\lambda r:\{x:\typ{int}, y:\typ{int}\}.r.x + r.y) \{x=5\}$$
			
			This application has type int according to our typing rules, but, much like
			the last example, will get stuck during evaluation even though it isn't a
			value.
			
\end{itemize}
\end{exercise}

\begin{exercise}
  Consider the simply-typed $\lambda$-calculus with records and
  subtyping.
%
\[
\begin{array}{rcl}
\tau & ::= & \{ l_1 \ty \tau_1, \dots, l_n \ty \tau_n \} \mid \tau_1 \arrow \tau_2\\
e    & ::= & x \mid e_1~e_2 \mid \lambda x \ty\tau .~ e \mid \{ l_1 = e_1 , \dots, l_n = e_n \} \mid e.l\\
\end{array}
\]
%
Prove progress and preservation. \\

\noindent \underline{Progress}: If $\vdash e:\tau$ then either $e$ is a value or $\exists e'. e \rightarrow e'$. \\

\begin{enumerate}[\hspace{20pt}]

\item 
\underline{Proof}: by induction on the typing derivation \\

Assume that we have some expression $e$ that has type $\tau$. To get that type 
judgement, we must have used one of our typing rules. \\


\underline{Case: T-Var} \\
This case is actually impossible--if you don't have a type context, its impossible to look up a variable and find its corresponding type, so T-Var couldn't have been used. \checkmark \checkmark \\

\underline{Case: T-App} \\
Here, we can use the form of the rule to conclude that 
\begin{itemize}
	\item $e = e_f e_{arg}$
	\item $e_f: \tau_{in} \rightarrow \tau_{out}$
	\item $e_{arg}: \tau_{in}$
\end{itemize}

And using our inductive hypothesis (i.e. progress) on $e_f$ and $e_arg$, we can conclude that because each expression is typed, they either are values or can step
forward to a new expression. This observation leads us to consider 3 cases:\\

\begin{enumerate}[\hspace{20pt}]
	\item
	\underline{Subcase: $e_f \rightarrow e_f'$} \\
	In this case, we can use the context
	rule to conclude that $e_f \text{ } e_{arg} \rightarrow e_f' \text{ } e_{arg}$. 
	In other words, the
	overall expression, $e \rightarrow e'$, where $e' = e_f' e_{arg}$. Progress holds
	in this case. \checkmark \\
	
	\underline{Subcase: $e_f$ is a value and $e_{arg} \rightarrow e_{arg}'$} \\
	Like the previous case $e = e_f \text{ } e_{arg} \rightarrow e_f \text{ } 
	e_{arg}'$, so $e$ can step forward again. \checkmark \\
	
	\underline{Subcase: $e_f$ and $e_{arg}$ are values} \\
	Here, we can use canonical forms to conclude that because $e_f$ has type 
	$\tau_{in} \rightarrow \tau_{out}$ and is value, it must be a lambda term. I.e.
	it has the form $\lambda x\ty\tau_{in}.e_{body}$, which means that $e_f
	\text{ }e_{arg}$ steps to
	$e_{body} \{e_{arg} / x\}$ by the $\beta$-reduction rule (which we realize looks 
	a little strange--it is valid, however, because $e_{arg}$ is really a value).
	\checkmark \\
\end{enumerate}

Regardless of whether $e_f$ and $e_arg$ are values or expressions that step, $e$ 
can still step forward $\checkmark \checkmark$ \\

\underline{Case: T-Abs} \\
Here, we can use the typing rule to infer that $e = \lambda x \ty 
\tau_in.e_{body} : \tau_{in} \rightarrow \tau_{out}$. So e is a value. \checkmark \checkmark \\

\underline{Case: T-RecordAccess} \\
Based off the form of the rule, we can conclude that $e = e_r.l_i$ and $e_r \ty 
\{l_1 \ty \tau_1, ... , l_n\ty\tau_n\}$. Applying the inductive hypothesis to $e_r$
we can conclude that the record expression can either step forward ($e_r 
\rightarrow e_r'$) or is already a value. In the first case, we can use the context 
rule to show that

$$e = e_r.l_i \rightarrow e_r'.l_i$$

So the expressions steps. And in the second case, we can apply the record access 
evaluation rule to conclude that $e = \{l_1 = v_1, ... \l_n = v_n\}.l_i \rightarrow v_i$. In either case, $e$ can take a step. \checkmark \checkmark \\

\underline{Case: T-RecordComposition} \\
Here, we know that $\vdash e = \{l_1 = e_1, ... , l_n = e_n \} \ty \{l_1 \ty 
\tau_1, ... l_n \ty \tau_n\}$ and $\forall i \in 1..n. \vdash e_i = \tau_i$. At
this point, either every sub-expression of the record is a value (in which case, 
the overall record is a value according to the syntax of our language), or there is 
a left-most sub-expression, $e_i$ that isn't a value. In the latter case, we can 
use the inductive hypothesis to assume that because $\vdash e_i \ty \tau_i$ it 
must either be a value or can take a step. Since $e_i$ isn't a value, $e_i 
\rightarrow e_i'$, and by context 

$$e= \{l_1 = v_1, ... , l_i = e_i, ... l_n = e_n \} \rightarrow \{l_1 = v_1, ... , 
l_i = e_i', ... l_n = e_n \}$$

Now its clear that in the latter case, $e$ can take a step forward, so in either 
case, progress holds. 
\checkmark \checkmark
\end{enumerate}

\noindent To prove preservation, we'll need the following lemma: \\

\noindent \underline{Lemma:} If $\Gamma \vdash e\ty\tau$ and $e = E[e']$ then $\exists \tau'. 
\Gamma \vdash e' \ty \tau'$ \\

\begin{enumerate}[\hspace{20pt}]
	\item
	\underline{Proof:} by structural induction on $E$ \\
	
	Assume $\Gamma \vdash e \ty \tau$ and $e = E[e']$. \\
	
	\underline{Case $E = [\cdot]$} \\
	In this case $e = e'$, and since we already have $\Gamma \vdash e \ty \tau$ we 
	know that $e'$ has type $\tau$ too. Consequently, $\exists \tau. \Gamma \vdash
	e' \ty \tau'$ ($\tau'$ is just $\tau$). \checkmark \\
	
	\underline{Case $E = E'\text{ } e_{arg}$} \\
	Here, $e = E[e'] = E'[e'] \text{ } e_{arg}$. Let's take a look at the derivation
	of $E'[e'] \ty \tau$ to see if we can get more information about $E'[e']$. As a
	side note, only T-App could have been used to produced this typing derivation,
	since it's the only rule whose conclusion can match application expressions:
	
	\begin{prooftree}
		\AxiomC{...}
		\UnaryInfC{$\Gamma \vdash E'[e'] \ty \tau_{in} \rightarrow \tau$}
		\AxiomC{...}
		\UnaryInfC{$\Gamma \vdash e_{arg} \ty \tau$}
		\BinaryInfC{$\Gamma \vdash E'[e'] e_{arg} \ty \tau$}
	\end{prooftree}
	
	Notice that $E'[e']$ is smaller than $E'[e'] \text{ } e_{arg}$, has a type, and 
	takes the form of an expression plugged into a context. We can 
	apply our inductive hypothesis (the lemma) to claim that $\exists \tau'. e' \ty 	\tau'$. \checkmark \\
	
	\underline{Case $E = v \text{ } E'$} \\
	Observe that $e = E[e'] = v E'[e']$ and that the typing derivation for $\Gamma 
	\vdash e \ty \tau$ implies $\Gamma \vdash E'[e'] \ty \tau$:
	
	\begin{prooftree}
		\AxiomC{...}
		\UnaryInfC{$\Gamma \vdash \ty \tau_{in} \rightarrow \tau$}
		\AxiomC{...}
		\UnaryInfC{$\Gamma \vdash E'[e'] \ty \tau$}
		\BinaryInfC{$\Gamma \vdash v E'[e'] \ty \tau$}
	\end{prooftree}
	
	Applying the inductive hypothesis to $E'[e']$, we can conclude that there is 
	$\tau'$ such that $\Gamma \vdash e' \ty \tau'$ \\
	
	\underline{Case $E = \{l_1 = v_1, ... , l_{i-1} = v_{i-1}, l_i = E', l_{i+1} = 
	e_{i+1}, ... , l_n = e_n\}$}\\
	Here, we can see
	$$e = E[e'] = \{..., l_i = E'[e'], ... \}$$
	\begin{prooftree}
		\AxiomC{...}
		\UnaryInfC{$\forall j \in 1 .. n. \Gamma \vdash e_j \ty \tau_k$}
		\UnaryInfC{$\Gamma \vdash \{l_1 = e_1, ..., l_i = E'[e'], ... l_n = e_n\} \ty \{l_1 \ty \tau_1, ... , l_i \ty \tau_i, ... , l_n \ty \tau_n\}$}
	\end{prooftree}
	
	Since $\forall j \in 1 .. n. \Gamma \vdash e_j \ty \tau_k$, we know that $\Gamma
	\vdash E'[e'] \ty \tau_i$. By the inductive hypothesis, $\exists \tau'. \Gamma 
	\vdash e' \ty \tau'$ \checkmark \\
	
	\underline{Case $E = E'.l_i$}
	In this case, $e = E[e'] = E'[e'].l_i$ and
	
	\begin{prooftree}
		\AxiomC{...}
		\UnaryInfC{$\Gamma \vdash E[e'] \ty \{l_1 \ty \tau_1, ... , l_n \ty \tau_n$} 
		\UnaryInfC{$\Gamma \vdash E'[e'].l_i \ty \tau_i = \tau$}
	\end{prooftree}
	
	And by the inductive hypothesis, $\exists \tau'. e':\tau$ \checkmark \\
\end{enumerate}

Using the structural induction principle, we can conclude that our lemma holds for
all contexts in this variant of simply-typed lambda calculus. \\

\noindent \underline{Preservation:} If $\vdash e \ty \tau$ and $e \rightarrow e'$ then $\vdash e' \ty \tau$ \\

\begin{enumerate}[\hspace{20pt}]
	\item
	\underline{Proof:} by induction on the derivation of $e \rightarrow e'$ \\
	
	We'll start this proof by assuming $\vdash e \ty \tau$ and $e \rightarrow e'$,
	and considering the possible final steps used in the derivation of 
	$e \rightarrow e'$: 
	
	\underline{Case: Record-access} \\
	Because the record access rule was used to derive $e \rightarrow e'$ we know that
	$e = \{l_1 = v_1, ... , l_n = v_n\}.l_i$ and $e' = v_i$. By the form of typing
	rules, we also know that the proof tree for $\vdash e \ty \tau$ must have
	had the following structure:
	
	\begin{prooftree}
		\AxiomC{...}
		\UnaryInfC{$\forall j \in 1 .. n. \vdash v_j \ty \tau_j$}
		\LeftLabel{T-RecordComposition}
		\UnaryInfC{$\vdash \{l_1 = v_1, ..., l_n = v_n\} \{l_1 \ty \tau_1, ... l_n \ty 		\tau_n\}$}
		\LeftLabel{T-RecordAccess}
		\UnaryInfC{$\vdash \{l_1 = v_1, ..., l_n = v_n \}.l_i \ty \tau_i = \tau $}
	\end{prooftree}
	
	Since we know $\forall j \in 1 .. n. \vdash v_j \ty \tau_j$, we know that $v_i
	\ty \tau_i$, which in turn implies that $e' = v_i \ty \tau_i = \tau$. \checkmark 
	\\
	
	
	\underline{Case Context} \\
	Matching off the evaluation rule for this case we know that $e = E[e_1]$,
	$e' = E[e_2]$ and $e_1 \rightarrow e_2$. Additionally, we can use the lemma
	introduced at the start of the proof to conclude that $e_1$ has some type $\tau'$
	and that, by the inductive hypothesis, $e_2$ shares that type. Finally, we can
	use the Context lemma to prove that because $E[e_1] \ty \tau$, $e_1 \ty \tau'$, 
	and $e_2 \ty \tau'$, then $E[e_2] \ty \tau$. \checkmark \\
	
	\underline{$\beta$-reduction}\\
	In this case, $e = (\lambda x \ty \tau_{in}. e_{body}) v$ and $e' = e_{body}\{v/x
	\}$.  Additionally, we know that the proof tree for the typing of e must have the
	following form:
	
	\begin{prooftree}
		\AxiomC{...}
		\UnaryInfC{$x \ty \tau_{in} \vdash e_{body} \ty \tau $}
		\LeftLabel{T-Abs}
		\UnaryInfC{$\vdash (\lambda x \ty \tau_{in}. e_{body}) \ty \tau_{in}
								\rightarrow \tau$}
		\AxiomC{...}
		\UnaryInfC{$\vdash v \ty \tau_{in}$}
		\LeftLabel{T-App}
		\BinaryInfC{$\vdash (\lambda x \ty \tau_{in}. e_{body}) v \ty \tau$}
	\end{prooftree}
	
	Using the substitution lemma on the facts in this proof tree (specifically,
	$x \ty \tau_{in} \vdash e_{body} \ty \tau $ and $\vdash v \ty \tau_{in}$) 
	we can conclude that $e' = e_{body}\{v/x\} \ty \tau$. \checkmark
\end{enumerate}
\end{exercise}

\begin{debriefing} \hfill\\[-4ex]
\begin{enumerate*}
\item How many hours did you spend on this assignment? 
\item Would you rate it as easy, moderate, or difficult? 
\item Did everyone in your study group participate? 
\item How deeply do you feel you understand the material it covers (0\%--100\%)? 
\item If you have any other comments, we would like to hear them!
  Please send email to \texttt{jnfoster@cs.cornell.edu}.
\end{enumerate*}
\end{debriefing}

\end{document}

