\newif\ifbeamer\beamerfalse
\documentclass[10pt]{article}

\usepackage{../tex/jnf}
\usepackage{alltt}

\renewcommand{\labelenumi}{\textbf{(\alph{enumi})}}

\begin{document}

Ameya Acharya (apa52) and Quinn Beightol (qeb2)\\

Homework 8\\


\begin{exercise}
\begin{enumerate}
\item   Suppose that we add a new typing rule to the simply-typed
  $\lambda$-calculus, extended with integers and unit:
\begin{center}
\infrule[T-Funny]
{}
{ \Gamma \vdash (\lambda x\!:\!\typ{unit}.~x) : \typ{int} }
{}
\end{center} 
Does progress still hold?  Does preservation still hold? If so,
explain why (briefly). If not, give a counterexample.

\item Now suppose instead that we add a new evaluation rule to the
simply-typed $\lambda$-calculus:
\begin{center}
\infrule[Funny2]
{}
{ e_1 + e_2 \stepsone () }
{}
\end{center}
Does progress still hold?  Does preservation still hold? If so,
explain why (briefly). If not, give a counterexample.
\end{enumerate}
\end{exercise}

\begin{exercise}
  The ``erasure'' of a System F term into a pure $\lambda$-calculus
  term can be defined as follows:
%
\[
\begin{array}{rcl}
\mathit{erase}(x) & = & x\\
\mathit{erase}(\lam{x \ty \tau}{e}) & = & \lam{x}{\mathit{erase}(e)}\\
\mathit{erase}(e_1~e_2) & = & (\mathit{erase}(e_1)~ \mathit{erase}(e_2))\\
\mathit{erase}(\Lam{X}{e}) & = & \lam{z}{\mathit{erase}(e)}~\qquad \text{where}~z~\text{fresh}\\
\mathit{erase}(e~[\tau]) & =& \mathit{erase}(e)~\lam{z}{z}
\end{array}
\]
%
Prove that if $e \arrow e'$ then
$\mathit{erase}(e) \Rightarrow \mathit{erase}(e')$ by induction (where
to avoid any confusion we let ``$\arrow$'' stand for the evaluation
relation in System F and ``$\Rightarrow$'' stand for the evaluation
relation for the pure $\lambda$-calculus). Although this property is
``obvious,'' proving it rigorously is still good practice!
\end{exercise}

\begin{exercise}
Consider System F extended with integers, booleans, and sums:
\begin{align*}
e    &::= x \mid \lam{x\ty\tau}{e} \mid e_1~e_2 \mid \Lam{X}{e} \mid e~[\tau] \mid n \mid e_1 + e_2  \\
     & \qquad \mid \lamfnt{true}  \mid \lamfnt{false} \mid e_1 = e_2 \mid \cond[\lamfnt]{e_1}{e_2}{e_3}  \\
     & \qquad \mid \inleft{\tau_1}{\tau_2}{e} \mid \inright{\tau_1}{\tau_2}{e} \mid \sumcase{e_1}{e_2}{e_3}\\
\tau &::= \typ{int} \mid \typ{bool} \mid \tau_1 \arrow \tau_2 \mid X \mid \forall X.~\tau \mid \tau_1 + \tau_2
\end{align*}
%
Two expressions $e_1$ and $e_2$ are said to be \emph{behaviorally
equivalent} if for any expression $f$, the expression $f~e_1$ behaves
the same as $f~e_2$. That is, it is impossible for $f$ to distinguish
$e_1$ from $e_2$. For example, the expressions
$\lam{x\ty\typ{unit}}{4+9}$ and $\lam{x\ty\typ{unit}}{13}$ are
behaviorally equivalent, since any function $f$, say
$\lam{t\ty\typ{unit}\arrow\typ{int}}{29+(t~())}$, behaves exactly the
same when given either one. However, the expressions
$\lam{x\ty\typ{int}}{0}$ and
$\lam{x\ty\typ{int}}{\cond[\lamfnt]{x=42}{1}{0}}$ are not behaviorally
equivalent because the expression
$\lam{y\ty{\typ{int}\arrow\typ{int}}}{y~42}$ distinguishes them.

For each of the following types, list the behaviorally \emph{distinct}
expressions of that type. That is, any expression that has that type
should be behaviorally equivalent to one of the expressions in your
list.  (Recall that all System F expressions terminate so you do not
need to consider non-termination as a possible behavior.)
%
\begin{enumerate}
\begin{minipage}{.5\textwidth}
\item $\forall A.~ A \rightarrow A$\\
\item $\forall A.~ A \rightarrow A \rightarrow A$\\
\item $\forall A.~\forall B.~ A \rightarrow (A \rightarrow B) \rightarrow B$\\
\item $\forall A.~\forall B.~ A \rightarrow B \rightarrow A + B$\\
\end{minipage}\begin{minipage}{.5\textwidth}
\item $\forall A.~ A$\\
\item $\forall A.~ A \rightarrow \typ{bool} \rightarrow A$\\
\item $\forall A.~ A \rightarrow A \rightarrow \typ{bool} \rightarrow A$\\
\bigskip\bigskip
\end{minipage}
\end{enumerate}
\end{exercise}

\begin{exercise}
  For each of the following, write ``Yes'' if it is okay to allow
  these types to be in the subtype relation or ``No'' if not. In
  addition, if your answer is ``No'' give a counterexample that shows
  how type soundness would break.
\begin{itemize}
\item $\typ{int} \subty \typ{unit}$
\item $\{ l : \top \} \subty \{ l : \typ{bool} \}$
\item $\{ \} \subty \{ x : \top \}$ 
\item $(\top \times \{ x \ty \typ{unit} \}) \subty (\{\} \times \top)$
\item $(\{ x \ty \typ{int} \} \arrow \typ{int}) \subty (\{ x \ty \typ{int}, y \ty \typ{int} \} \arrow \typ{int})$
\item $(\{ x \ty \typ{int}, y \ty \typ{int} \} \arrow \typ{int}) \subty (\{ x \ty \typ{int} \} \arrow \typ{int})$
\end{itemize}
\end{exercise}

\begin{exercise}
  Consider the simply-typed $\lambda$-calculus with records and
  subtyping.
%
\[
\begin{array}{rcl}
\tau & ::= & \{ l_1 \ty \tau_1, \dots, l_n \ty \tau_n \} \mid \tau_1 \arrow \tau_2\\
e    & ::= & x \mid e_1~e_2 \mid \lambda x \ty\tau .~ e \mid \{ l_1 = e_1 , \dots, l_n = e_n \} \mid e.l\\
\end{array}
\]
%
Prove progress and preservation. 
\end{exercise}

\begin{debriefing} \hfill\\[-4ex]
\begin{enumerate*}
\item How many hours did you spend on this assignment? 
\item Would you rate it as easy, moderate, or difficult? 
\item Did everyone in your study group participate? 
\item How deeply do you feel you understand the material it covers (0\%--100\%)? 
\item If you have any other comments, we would like to hear them!
  Please send email to \texttt{jnfoster@cs.cornell.edu}.
\end{enumerate*}
\end{debriefing}

\end{document}

