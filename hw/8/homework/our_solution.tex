\newif\ifbeamer\beamerfalse
\documentclass[10pt]{article}

\usepackage{../tex/jnf}
\usepackage{alltt}

\renewcommand{\labelenumi}{\textbf{(\alph{enumi})}}

\begin{document}

Ameya Acharya (apa52) and Quinn Beightol (qeb2)\\

Homework 8\\


\begin{exercise}
\begin{enumerate}
\item   Progress does not hold. Progress states that if $\vdash e: \tau$ then e is either a value or $\exists e'. e \rightarrow e'.$ Consider ($\lambda x: unit \text{ } x) + 5$. This is not a value. So, we consider if it can step. We see that the expression is well-typed, because$\lambda x: unit \text{ } x$ has type $int$ according to our new rule. However, there is no value to which this steps. Therefore, our new rule breaks progress. \\

Preservation holds. Preservation states that if $\vdash e: \tau$ and $e \rightarrow e'$ then $\vdash e': \tau$. We are changing the typing rule for a value, which doesn't step. 

\item  Progress holds: we still have all the old evaluation rules and we haven't changed the type rules. Therefore, we can just re-use the old evaluation rules if necessary.\\

Preservation breaks: consider $(4 + 5) : int$. Using the new rules, we have $(4 + 5) \rightarrow ()$, and $() : unit.$ Under preservation, if $4 + 5$ steps, it is expected to step to an expression of type $int,$ which we do not have. Therefore, preservation breaks.
 
\end{enumerate}
\end{exercise}

\begin{exercise}
%  The ``erasure'' of a System F term into a pure $\lambda$-calculus
  %term can be defined as follows:
%
%\[
%\begin{array}{rcl}
%\mathit{erase}(x) & = & x\\
%\mathit{erase}(\lam{x \ty \tau}{e}) & = & \lam{x}{\mathit{erase}(e)}\\
%\mathit{erase}(e_1~e_2) & = & (\mathit{erase}(e_1)~ \mathit{erase}(e_2))\\
%\mathit{erase}(\Lam{X}{e}) & = & \lam{z}{\mathit{erase}(e)}~\qquad \text{where}~z~\text{fresh}\\
%\mathit{erase}(e~[\tau]) & =& \mathit{erase}(e)~\lam{z}{z}
%\end{array}
%\]
%
We prove that if $e \arrow e'$ then
$\mathit{erase}(e) \Rightarrow \mathit{erase}(e')$ by structural induction on e.\\

First, we prove the following lemma: $\mathit{erase}(e\{v/x\}) = \mathit{erase}(e)\{v/x\}$.\\
We proceed by structural induction on $e.$\\

{\sc case} $e = y$:
\begin{itemize}
\item $\mathit{erase}(y \{v/x\}) = \mathit{erase}(y) = y.$
\item $\mathit{erase}(y) \{v/x\} = (y)\{v/x\} = y.$\\

\checkmark
\end{itemize}

{\sc case} $e = \lambda y: \tau. e_{body}$:
\begin{itemize}
\item $\mathit{erase}(\lambda y: \tau. e_{body} \{v/x\}) = \mathit{erase}(\lambda y: \tau. e_{body})\{v/x\}$ 
= $\lambda y. \mathit{erase}(e_{body} \{v/x\}).$\\
\item $\mathit{erase}(\lambda y: \tau. e_{body}) \{v/x\} = (\lambda y: \tau. \mathit{erase} (e_{body})) \{v/x\} = \lambda y. \mathit{erase} (e_{body}) \{v/x\}.$\\

\end{itemize}

By the inductive hypothesis, $\mathit{erase}(e_{body}\{v/x\}) = \mathit{erase}(e_{body})\{v/x\}$. Therefore, our two expressions above are equal, which concludes this case. \checkmark\\

{\sc case} $e = e_1e_2$:
\begin{itemize}
\item $\mathit{erase}((e_1e_2)\{v/x\}) = \mathit{erase}(e_1\{v/x\} \text{ } e_2\{v/x\}) =  \mathit{erase}(e_1\{v/x\})\text{ } \mathit{erase}(e_2\{v/x\})$
\item $\mathit{erase}(e_1e_2)\{v/x\} =  [\mathit{erase}(e_1)\text{ } \mathit{erase}(e_2)] \{v/x\} = \mathit{erase}(e_1)\{v/x\} \text{ } \mathit{erase}(e_2)\{v/x\}$\\
\end{itemize}

By the inductive hypothesis, $\mathit{erase}(e_1\{v/x\}) = \mathit{erase}(e_1)\{v/x\}$. Therefore, our two expressions above are equal, which concludes this case. \checkmark\\

{\sc case}$e = \Lambda X.e_{body} $: TODO.\\

{\sc case}$e = e_{poly}[\tau] $: TODO.\\



Now, we return to our original proof: $e \arrow e'$ then
$\mathit{erase}(e) \Rightarrow \mathit{erase}(e')$.\\

{\sc case} $\beta$ {\sc -reduction}: TODO.\\

{\sc case type-reduction}: TODO.\\

{\sc case context}: TODO.\\


  

%(where
%to avoid any confusion we let ``$\arrow$'' stand for the evaluation
%relation in System F and ``$\Rightarrow$'' stand for the evaluation
%relation for the pure $\lambda$-calculus). Although this property is
%``obvious,'' proving it rigorously is still good practice!
\end{exercise}

\begin{exercise}
\begin{enumerate}
\begin{minipage}{.5\textwidth}
\item  $\Lambda A. \lambda x:A. x$   \\
\item $\Lambda A. \lambda x:A. \lambda y:A. x$\\
$\Lambda A. \lambda x:A. \lambda y:A. y$
\item $\Lambda A. \Lambda B. \lambda x:A. \lambda f:A \rightarrow B. \text{ } f x$ \\
\item $\Lambda A. \Lambda B. \lambda x:A. \lambda y:B. \text{inl } x$\\
$\Lambda A. \Lambda B. \lambda x:A. \lambda y:B. \text{inr } y$
\end{minipage}\begin{minipage}{.5\textwidth}
\item Nothing\\
\item $\Lambda A. \lambda x:A. \lambda b:\text{bool}. x $\\
\item  $\Lambda A. \lambda x:A. \lambda y:A. \lambda b:\text{bool}. x $\\
$\Lambda A. \lambda x:A. \lambda y:A. \lambda b:\text{bool}. y $\\
$\Lambda A. \lambda x:A. \lambda y:A. \lambda b:\text{bool}. \text{if } b \text{ then } x \text{ else } y $\\
$\Lambda A. \lambda x:A. \lambda y:A. \lambda b:\text{bool}. \text{if } b \text{ then } y \text{ else } x$\\
\bigskip\bigskip
\end{minipage}
\end{enumerate}
\end{exercise}

\begin{exercise}
  For each of the following, write ``Yes'' if it is okay to allow
  these types to be in the subtype relation or ``No'' if not. In
  addition, if your answer is ``No'' give a counterexample that shows
  how type soundness would break.
\begin{itemize}
\item $\typ{int} \subty \typ{unit}$
\item $\{ l : \top \} \subty \{ l : \typ{bool} \}$
\item $\{ \} \subty \{ x : \top \}$ 
\item $(\top \times \{ x \ty \typ{unit} \}) \subty (\{\} \times \top)$
\item $(\{ x \ty \typ{int} \} \arrow \typ{int}) \subty (\{ x \ty \typ{int}, y \ty \typ{int} \} \arrow \typ{int})$
\item $(\{ x \ty \typ{int}, y \ty \typ{int} \} \arrow \typ{int}) \subty (\{ x \ty \typ{int} \} \arrow \typ{int})$
\end{itemize}
\end{exercise}

\begin{exercise}
  Consider the simply-typed $\lambda$-calculus with records and
  subtyping.
%
\[
\begin{array}{rcl}
\tau & ::= & \{ l_1 \ty \tau_1, \dots, l_n \ty \tau_n \} \mid \tau_1 \arrow \tau_2\\
e    & ::= & x \mid e_1~e_2 \mid \lambda x \ty\tau .~ e \mid \{ l_1 = e_1 , \dots, l_n = e_n \} \mid e.l\\
\end{array}
\]
%
Prove progress and preservation. 
\end{exercise}

\begin{debriefing} \hfill\\[-4ex]
\begin{enumerate*}
\item How many hours did you spend on this assignment? 
\item Would you rate it as easy, moderate, or difficult? 
\item Did everyone in your study group participate? 
\item How deeply do you feel you understand the material it covers (0\%--100\%)? 
\item If you have any other comments, we would like to hear them!
  Please send email to \texttt{jnfoster@cs.cornell.edu}.
\end{enumerate*}
\end{debriefing}

\end{document}

