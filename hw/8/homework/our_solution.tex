\newif\ifbeamer\beamerfalse
\documentclass[10pt]{article}

\usepackage{../tex/jnf}
\usepackage{alltt}

\renewcommand{\labelenumi}{\textbf{(\alph{enumi})}}

\begin{document}

Ameya Acharya (apa52) and Quinn Beightol (qeb2)\\

Homework 8\\


\begin{exercise}
\begin{enumerate}
\item   Progress does not hold: (consider ($\lambda x: unit x) + 5).\\

Preservation is 

\item Now suppose instead that we add a new evaluation rule to the
simply-typed $\lambda$-calculus:
\begin{center}
\infrule[Funny2]
{}
{ e_1 + e_2 \stepsone () }
{}
\end{center}
Does progress still hold?  Does preservation still hold? If so,
explain why (briefly). If not, give a counterexample.
\end{enumerate}
\end{exercise}

\begin{exercise}
  The ``erasure'' of a System F term into a pure $\lambda$-calculus
  term can be defined as follows:
%
\[
\begin{array}{rcl}
\mathit{erase}(x) & = & x\\
\mathit{erase}(\lam{x \ty \tau}{e}) & = & \lam{x}{\mathit{erase}(e)}\\
\mathit{erase}(e_1~e_2) & = & (\mathit{erase}(e_1)~ \mathit{erase}(e_2))\\
\mathit{erase}(\Lam{X}{e}) & = & \lam{z}{\mathit{erase}(e)}~\qquad \text{where}~z~\text{fresh}\\
\mathit{erase}(e~[\tau]) & =& \mathit{erase}(e)~\lam{z}{z}
\end{array}
\]
%
Prove that if $e \arrow e'$ then
$\mathit{erase}(e) \Rightarrow \mathit{erase}(e')$ by induction (where
to avoid any confusion we let ``$\arrow$'' stand for the evaluation
relation in System F and ``$\Rightarrow$'' stand for the evaluation
relation for the pure $\lambda$-calculus). Although this property is
``obvious,'' proving it rigorously is still good practice!
\end{exercise}

\begin{exercise}
\begin{enumerate}
\begin{minipage}{.5\textwidth}
\item  $\Lambda A. \lambda x:A. x$   \\
\item $\Lambda A. \lambda x:A. \lambda y:A. x$\\
$\Lambda A. \lambda x:A. \lambda y:A. y$
\item $\Lambda A. \Lambda B. \lambda x:A. \lambda f:A \rightarrow B. \text{ } f x$ \\
\item $\Lambda A. \Lambda B. \lambda x:A. \lambda y:B. \text{inl } x$\\
$\Lambda A. \Lambda B. \lambda x:A. \lambda y:B. \text{inr } y$
\end{minipage}\begin{minipage}{.5\textwidth}
\item Nothing\\
\item $\Lambda A. \lambda x:A. \lambda b:\text{bool}. x $\\
\item  $\Lambda A. \lambda x:A. \lambda y:A. \lambda b:\text{bool}. x $\\
$\Lambda A. \lambda x:A. \lambda y:A. \lambda b:\text{bool}. y $\\
$\Lambda A. \lambda x:A. \lambda y:A. \lambda b:\text{bool}. \text{if } b \text{ then } x \text{ else } y $\\
$\Lambda A. \lambda x:A. \lambda y:A. \lambda b:\text{bool}. \text{if } b \text{ then } y \text{ else } x$\\
\bigskip\bigskip
\end{minipage}
\end{enumerate}
\end{exercise}

\begin{exercise}
  For each of the following, write ``Yes'' if it is okay to allow
  these types to be in the subtype relation or ``No'' if not. In
  addition, if your answer is ``No'' give a counterexample that shows
  how type soundness would break.
\begin{itemize}
\item $\typ{int} \subty \typ{unit}$
\item $\{ l : \top \} \subty \{ l : \typ{bool} \}$
\item $\{ \} \subty \{ x : \top \}$ 
\item $(\top \times \{ x \ty \typ{unit} \}) \subty (\{\} \times \top)$
\item $(\{ x \ty \typ{int} \} \arrow \typ{int}) \subty (\{ x \ty \typ{int}, y \ty \typ{int} \} \arrow \typ{int})$
\item $(\{ x \ty \typ{int}, y \ty \typ{int} \} \arrow \typ{int}) \subty (\{ x \ty \typ{int} \} \arrow \typ{int})$
\end{itemize}
\end{exercise}

\begin{exercise}
  Consider the simply-typed $\lambda$-calculus with records and
  subtyping.
%
\[
\begin{array}{rcl}
\tau & ::= & \{ l_1 \ty \tau_1, \dots, l_n \ty \tau_n \} \mid \tau_1 \arrow \tau_2\\
e    & ::= & x \mid e_1~e_2 \mid \lambda x \ty\tau .~ e \mid \{ l_1 = e_1 , \dots, l_n = e_n \} \mid e.l\\
\end{array}
\]
%
Prove progress and preservation. 
\end{exercise}

\begin{debriefing} \hfill\\[-4ex]
\begin{enumerate*}
\item How many hours did you spend on this assignment? 
\item Would you rate it as easy, moderate, or difficult? 
\item Did everyone in your study group participate? 
\item How deeply do you feel you understand the material it covers (0\%--100\%)? 
\item If you have any other comments, we would like to hear them!
  Please send email to \texttt{jnfoster@cs.cornell.edu}.
\end{enumerate*}
\end{debriefing}

\end{document}

