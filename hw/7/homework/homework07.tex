\newif\ifbeamer\beamerfalse
\documentclass[10pt]{article}

\usepackage{../tex/jnf}
\usepackage{alltt}

\renewcommand{\labelenumi}{\textbf{(\alph{enumi})}}

\begin{document}

\bigredheader{Homework \#7}

\paragraph{Due} Wedneday, November 5, 2014 by 11:59pm 

\paragraph{Instructions} 
This assignment may be completed with one partner.  You and your
partner should submit a single solution on CMS. Please do not offer or
accept any other assistance on this assignment. Late submissions will
not be accepted.

\begin{exercise}
Your task in this assignment is to implement an interpreter for the
$\lambda$-calculus extended with unit, integers, booleans, and pairs:
%
\begin{alltt}
type exp =
    Var of var
  | App of exp * exp
  | Lam of var * exp
  | Let of var * exp * exp
  | Unit 
  | Int of int
  | Plus of exp * exp
  | Pair of exp * exp
  | Fst of exp
  | Snd of exp
  | True
  | False
  | Eq of exp * exp
  | If of exp * exp * exp
\end{alltt}
%
However, there's a twist: before evaluating the expression, we want to
you first transform it into continuation-passing style (see Lecture
18) so that it has the following form:
%
\begin{alltt}
type cps{\char95}exp = 
    CApp of cps{\char95}exp * cps{\char95}atom
  | CAtom of cps{\char95}atom 

and cps{\char95}atom = 
    CVar of var
  | CLam of var * cps{\char95}exp
  | CUnit 
  | CInt of int
  | CPlus of var * var 
  | CPair of var * var
  | CFst of var
  | CSnd of var
  | CTrue
  | CFalse
  | CEq of var * var
  | CIf of var * var * var
\end{alltt} 
Intuitively, a CPS expression (\texttt{cps{\char95}exp}) represents
either a single CPS atom (\texttt{cps{\char95}atom}), or an
application of atoms. A CPS atom is either a variable, a value, a data
constructor applied to zero or more variables
(e.g., \texttt{CPair(x,y)}, or a primitive function applied to zero or
more variables (e.g., \texttt{CFst(x)}). To put an arbitrary
expression into this form, you will have to instrument the program
with continuations, which will have the effect of making the control
flow explicit.

Then, to finish your interpreter, you should write a large-step
interpreter that evaluates a CPS expression to a value:
\begin{alltt}
and cps_val = 
  | VUnit 
  | VInt of int
  | VClosure of cps_env * var * cps_exp 
  | VPair of cps_val * cps_val
  | VTrue 
  | VFalse
\end{alltt}  
A closure \texttt{VClosure(g,x,e)} reprsents a function with
argument \texttt{x} and body \texttt{e}. The environment \texttt{g}
gives the bindings for any free variables
in \texttt{e}. The \texttt{Ast} module defines a representation of
environments and accompanying operations on them.

To help you get started, we have provided some files in a tarball on
CMS. The only files you need to submit are \texttt{cps.ml}
and \texttt{eval.ml}.

Some additional notes:
\begin{itemize}
\item Your CPS translation and interpreter should implement  
left-to-right, call-by-value semantics.
\item It will not be possible to evaluate certain expressions---e.g., 
the application \texttt{(3 5)}. Your interpreter should raise
the \texttt{IllformedExpression} exception on such inputs.
\item The \texttt{Fvs} module contains a number of useful definitions 
for calculating free variables and obtaining fresh variables. In
particular, the function \texttt{fresh} generates a variable like its
first argument that is guaranteed to be fresh for its second
argument. For example, evaluating \texttt{fresh "y" (fvs (App(Var "x",
Var "y")))} returns \texttt{"y\_0"}.
\end{itemize}
\end{exercise}

\begin{exercise}
Develop a suite of $10$ or more distinct tests, each contained in a
separate file. Your tests should execute successfully and terminate
with a value. We will give extra credit for any tests that expose a
bug in our interpreter. Please submit a single archive
\texttt{tests.tar.gz} containing the tests in files \texttt{test-01.lam}
through \texttt{test-10.lam}.
\end{exercise}

\bigskip

\noindent \textbf{Karma Extensions:}
\begin{itemize}
\item Modify your interpreter (but not the CPS translation) to have call-by-name semantics.
\item Optimize your CPS translation to remove extra ``administrative'' $\lambda$-abstractions. 
\item Extend the langauge with control constructs such as exceptions or call/cc.
\end{itemize}

\begin{debriefing} \hfill\\[-4ex]
\begin{enumerate*}
\item How many hours did you spend on this assignment? 
\item Would you rate it as easy, moderate, or difficult? 
\item Did everyone in your study group participate? 
\item How deeply do you feel you understand the material it covers (0\%--100\%)? 
\item If you have any other comments, we would like to hear them!
  Please send email to \texttt{jnfoster@cs.cornell.edu}.
\end{enumerate*}
\end{debriefing}

\end{document}

